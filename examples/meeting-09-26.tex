\documentclass[11pt,letterpaper]{article}
\usepackage[utf8]{inputenc}
\usepackage{caption} % for table captions
\usepackage{amsmath} % for multi-line equations and piecewises
\DeclareMathOperator{\sign}{sign}
\usepackage{graphicx}
\usepackage{relsize}
\usepackage{xspace}
\usepackage{float} 
\usepackage{verbatim} % for block comments
\usepackage{subcaption} % for subfigures
\usepackage{enumitem} % for a) b) c) lists
\newcommand{\Cyclus}{\textsc{Cyclus}\xspace}%
\newcommand{\Cycamore}{\textsc{Cycamore}\xspace}%
\newcommand{\deploy}{\texttt{d3ploy}\xspace}%
\newcommand{\Deploy}{\texttt{D3ploy}\xspace}%
\usepackage{tabularx}
\usepackage{color}
\usepackage{multirow}
\usepackage[acronym,toc]{glossaries}
\include{acros}
\definecolor{bg}{rgb}{0.95,0.95,0.95}
\newcolumntype{b}{X}
\newcolumntype{f}{>{\hsize=.15\hsize}X}
\newcolumntype{s}{>{\hsize=.5\hsize}X}
\newcolumntype{m}{>{\hsize=.75\hsize}X}
\newcolumntype{r}{>{\hsize=1.1\hsize}X}
\usepackage{titling}
\usepackage[hang,flushmargin]{footmisc}
\renewcommand*\footnoterule{}
\usepackage{tikz}

\usetikzlibrary{shapes.geometric,arrows}
\tikzstyle{process} = [rectangle, rounded corners, 
minimum width=1cm, minimum height=1cm,text centered, draw=black, 
fill=blue!30]
\tikzstyle{arrow} = [thick,->,>=stealth]

\graphicspath{}
%\title{Thermo-Mechanical MOOSE-based solver}
%\author{Roberto E. Fairhurst Agosta}

\begin{document}
	%\begin{titlepage}
		%\maketitle
		%\thispagestyle{empty}
	%\end{titlepage}
	
\section{Meeting 09/26}

References:
\begin{description}[font=$\bullet$\scshape\bfseries]
	\item[] $\sigma_i$: stress in $i$-direction
	\item[] $\tau_{ij}$: shear stress normal to $i$, in $j$-direction
	\item[] $\rho$: density
	\item[] $g$: gravity
	\item[] $\varepsilon_i$: strain in $i$-direction
	\item[] $\gamma_{ij}$: shear strain
	\item[] $E$: young's modulus
	\item[] $\nu$: Poisson ratio
	\item[] $u$: displacement in x-direction
	\item[] $v$: displacement in y-direction
\end{description}

2-D case. Plane Stress. It assumes that $\sigma_z = 0$, $\tau_{xz} = 0$, $\tau_{yz} = 0$.

Equilibrium eqs:
\begin{equation}
\frac{\partial \sigma_x}{\partial x} + \frac{\partial \tau_{xy}}{\partial y} = 0
\end{equation}
\begin{equation}
\frac{\partial \tau_{xy}}{\partial x} + \frac{\partial \sigma_y}{\partial y} + \rho g = 0
\end{equation}

Hooke's Law:
\begin{equation}
\varepsilon_x = \frac{1}{E} (\sigma_x - \nu \sigma_y)
\end{equation}
\begin{equation}
\varepsilon_y = \frac{1}{E} (\sigma_y - \nu \sigma_x)
\end{equation}
\begin{equation}
\gamma_{xy} = \frac{2(1+\nu)}{E} \tau_{xy}
\label{eq1}
\end{equation}

Deformation eqs:
\begin{equation}
\varepsilon_x = \frac{\partial u}{\partial x}
\end{equation}
\begin{equation}
\varepsilon_y = \frac{\partial v}{\partial y}
\end{equation}
\begin{equation}
\gamma_{xy} = \frac{\partial v}{\partial x} + \frac{\partial u}{\partial y}
\label{eq2}
\end{equation}

Putting together \ref{eq1} and \ref{eq2}:
\begin{equation}
\tau_{xy} = \frac{E}{2(1+\nu)} (\frac{\partial v}{\partial x} + \frac{\partial u}{\partial y})
\label{eq3}
\end{equation}

The equilibrium eqs and the deformation equations take the \textit{Kernel} form:
\begin{equation}
\left< \frac{\partial \sigma_x}{\partial x}, \phi \right> + \left< \frac{\partial \tau_{xy}}{\partial y}, \phi \right> = 0
\end{equation}
\begin{equation}
\left< \frac{\partial \tau_{xy}}{\partial x}, \phi \right> + \left< \frac{\partial \sigma_{y}}{\partial y}, \phi \right> + \left< \rho g , \phi \right>= 0
\end{equation}
\begin{equation}
\left< \varepsilon_x, \phi \right> - \left< \frac{\partial u}{\partial x}, \phi \right> = 0
\end{equation}
\begin{equation}
\left< \varepsilon_y, \phi \right> - \left< \frac{\partial v}{\partial y}, \phi \right> = 0
\end{equation}

The hooke's equations and equation \ref{eq3} take the form of \textit{AuxKernels}.

Eqs with their respective implementation in code lines:

\begin{equation}
\left< \frac{\partial \sigma_x}{\partial x}, \phi \right> + \left< \frac{\partial \tau_{xy}}{\partial y}, \phi \right> = 0
\label{eq:eq_x}
\end{equation}
\begin{equation}
\left< \frac{\partial \tau_{xy}}{\partial x}, \phi \right> + \left< \frac{\partial \sigma_{y}}{\partial y}, \phi \right> + \left< \rho g , \phi \right>= 0
\label{eq:eq_y}
\end{equation}
\begin{figure}[H]
	\centering
	\includegraphics[width=1.1\linewidth]{st09/eq.png}
	\hfill
	\caption{Equations \ref{eq:eq_x} and \ref{eq:eq_y}.}
	\label{fig:eq_xy}
\end{figure}

\begin{equation}
\left< \varepsilon_x, \phi \right> - \left< \frac{\partial u}{\partial x}, \phi \right> = 0
\label{eq:def_x}
\end{equation}
\begin{equation}
\left< \varepsilon_y, \phi \right> - \left< \frac{\partial v}{\partial y}, \phi \right> = 0
\label{eq:def_y}
\end{equation}
\begin{figure}[H]
	\centering
	\includegraphics[width=1.1\linewidth]{st09/def.png}
	\hfill
	\caption{Equations \ref{eq:def_x} and \ref{eq:def_y}.}
	\label{fig:def_xy}
\end{figure}

\begin{equation}
\varepsilon_x = \frac{1}{E} (\sigma_x - \nu \sigma_y)
\label{cons1}
\end{equation}
\begin{equation}
\varepsilon_y = \frac{1}{E} (\sigma_y - \nu \sigma_x)
\label{cons2}
\end{equation}
\begin{figure}[H]
	\centering
	\includegraphics[width=1.1\linewidth]{st09/cons.png}
	\hfill
	\caption{Equations \ref{cons1} and \ref{cons2}.}
	\label{fig:cons_xy}
\end{figure}

\begin{equation}
\tau_{xy} = \frac{E}{2(1+\nu)} (\frac{\partial v}{\partial x} + \frac{\partial u}{\partial y})
\end{equation}
\begin{figure}[H]
	\centering
	\includegraphics[width=1.1\linewidth]{st09/shear.png}
	\hfill
	\caption{Equations \ref{eq3}.}
	\label{fig:shear}
\end{figure}

Input file:

BCs: 
\begin{figure}[H]
	\centering
	\includegraphics[width=0.7\linewidth]{st09/bcs.png}
	\hfill
	\caption{BCs.}
	\label{fig:bcs}
\end{figure}

\pagebreak 
\bibliographystyle{plain}
%\bibliography{bibliography}

\end{document}
