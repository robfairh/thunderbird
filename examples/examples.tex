\documentclass[11pt,letterpaper]{article}
\usepackage[utf8]{inputenc}
\usepackage{caption} % for table captions
\usepackage{amsmath} % for multi-line equations and piecewises
\DeclareMathOperator{\sign}{sign}
\usepackage{graphicx}
\usepackage{relsize}
\usepackage{xspace}
\usepackage{verbatim} % for block comments
\usepackage{subcaption} % for subfigures
\usepackage{enumitem} % for a) b) c) lists
\newcommand{\Cyclus}{\textsc{Cyclus}\xspace}%
\newcommand{\Cycamore}{\textsc{Cycamore}\xspace}%
\newcommand{\deploy}{\texttt{d3ploy}\xspace}%
\newcommand{\Deploy}{\texttt{D3ploy}\xspace}%
\usepackage{tabularx}
\usepackage{color}
\usepackage{multirow}
\usepackage[acronym,toc]{glossaries}
\include{acros}
\definecolor{bg}{rgb}{0.95,0.95,0.95}
\newcolumntype{b}{X}
\newcolumntype{f}{>{\hsize=.15\hsize}X}
\newcolumntype{s}{>{\hsize=.5\hsize}X}
\newcolumntype{m}{>{\hsize=.75\hsize}X}
\newcolumntype{r}{>{\hsize=1.1\hsize}X}
\usepackage{titling}
\usepackage[hang,flushmargin]{footmisc}
\renewcommand*\footnoterule{}
\usepackage{tikz}

\usetikzlibrary{shapes.geometric,arrows}
\tikzstyle{process} = [rectangle, rounded corners, 
minimum width=1cm, minimum height=1cm,text centered, draw=black, 
fill=blue!30]
\tikzstyle{arrow} = [thick,->,>=stealth]

\graphicspath{}
\title{An Introduction to Moltres}
\author{Roberto E. Fairhurst Agosta}

\begin{document}
	\begin{titlepage}
		\maketitle
		\thispagestyle{empty}
	\end{titlepage}
	
\section{Example 00}

Kernels involved:
\begin{description}[font=$\bullet$\scshape\bfseries]
	\item[] moose/Diffusion
\end{description}

Equations involved:
\begin{equation}
-\nabla.\nabla T=0
\end{equation}

Boundary conditions:
\begin{description}[]
	\item[] $T('left')=0$ and $T('right')=1$
\end{description}

The results are presented in Figure \ref{fig:exa00}.
\begin{figure*}[!h]
	\centering
	\includegraphics[width=\linewidth]{00/rectangularmesh.png} 
	\hfill
	\caption{Temperature.}
	\label{fig:ex00}
\end{figure*}

\newpage
\section{Example 01}

Kernels involved:
\begin{description}[font=$\bullet$\scshape\bfseries]
	\item[] moose/Diffusion
	\item[] Convection
\end{description}

Equations involved:
\begin{equation}
-\nabla.\nabla T + \vec{v}\nabla T=0
\end{equation}

Boundary conditions:
\begin{description}[]
	\item[] $T('bottom')=0$, $T('left')=1$, and $T('right')=1$
\end{description}

The results are presented in Figure \ref{fig:exa01}.
\begin{figure*}[!h]
	\centering
	\includegraphics[width=\linewidth]{01/temperature.png} 
	\hfill
	\caption{Temperature.}
	\label{fig:exa01}
\end{figure*}

\newpage
\section{Example 02}

Kernels involved:
\begin{description}[font=$\bullet$\scshape\bfseries]
	\item[] NtDiffusion
	\item[] NtSource
\end{description}

Equations involved:
\begin{equation}
-\nabla.(D\nabla \phi) - S=0
\end{equation}

Boundary conditions:
\begin{description}[]
	\item[] $\phi('left')=0$, and $\phi('right')=0$
\end{description}

The results are presented in Figure \ref{fig:exa02}.
\begin{figure*}[!h]
	\centering
	\includegraphics[width=\linewidth]{02/constantsource.png} 
	\hfill
	\caption{Flux.}
	\label{fig:exa02}
\end{figure*}

\newpage
\section{Example 03}

Kernels involved:
\begin{description}[font=$\bullet$\scshape\bfseries]
	\item[] NtDiffusion
	\item[] NtSigmaA
\end{description}

Equations involved:
\begin{equation}
-\nabla.(D\nabla \phi) + \Sigma_{A}\phi =0
\end{equation}

Boundary conditions:
\begin{description}[]
	\item[] $\phi('left')=0$, and $\phi('right')=1$
\end{description}

The results are presented in Figure \ref{fig:exa03}.
\begin{figure*}[!h]
	\centering
	\includegraphics[width=\linewidth]{03/absorption.png} 
	\hfill
	\caption{Flux.}
	\label{fig:exa03}
\end{figure*}

\newpage
\section{Example 04}

Kernels involved:
\begin{description}[font=$\bullet$\scshape\bfseries]
	\item[] NtDiffusion
	\item[] NtSigmaF
\end{description}

Equations involved:
\begin{equation}
-\nabla.(D\nabla \phi) - \nu\Sigma_{F}\phi =0
\end{equation}

Boundary conditions:
\begin{description}[]
	\item[] $\phi('left')=0$, and $\phi('right')=1$
\end{description}

The results are presented in Figure \ref{fig:exa04}.
\begin{figure*}[!h]
	\centering
	\includegraphics[width=\linewidth]{04/fission.png} 
	\hfill
	\caption{Flux.}
	\label{fig:exa04}
\end{figure*}

\newpage
\section{Example 05}

Kernels involved:
\begin{description}[font=$\bullet$\scshape\bfseries]
	\item[] NtTimeDerivative
	\item[] NtDiffusion
\end{description}

Equations involved:
\begin{equation}
a\frac{\partial\phi}{\partial t}-\nabla.(D\nabla \phi)=0
\end{equation}

Boundary conditions:
\begin{description}[]
	\item[] $\phi('bottom')=0$, and $\phi('top')=1$
\end{description}

The results are presented in Figure \ref{fig:exa05}.
\begin{figure*}[!h]
	\centering
	\begin{subfigure}[t]{0.4\textwidth}
		\centering
		\includegraphics[width=\linewidth]{05/timederivative_t=0.png} 
		\caption{$\phi$ at t=0.}
		\label{fig:exa01-flux1}
	\end{subfigure}
	\vspace{1cm}
	\begin{subfigure}[t]{0.4\textwidth}
		\centering
		\includegraphics[width=\linewidth]{05/timederivative_t=50.png} 
		\caption{$\phi$ at t=50.}
		\label{fig:exa01-flux2}
	\end{subfigure}
	\hfill
	\caption{Flux.}
	\label{fig:exa05}
\end{figure*}

\newpage
\section{Example 06}

Kernels involved:
\begin{description}[font=$\bullet$\scshape\bfseries]
	\item[] NtDiffusion
	\item[] InScatter
	\item[] NtSource
\end{description}

Equations involved:
\begin{equation}
-\nabla.(D_{1}\nabla \phi_{1})-S=0
\end{equation}
\begin{equation}
-\nabla.(D_{2}\nabla \phi_{2})-\Sigma_{S1-2}\phi_{1}=0
\end{equation}

Boundary conditions:
\begin{description}[]
	\item[] $\phi_{1}('left')=0$, and $\phi_{1}('right')=0$
	\item[] $\phi_{2}('left')=0$, and $\phi_{2}('right')=0$
\end{description}

The results are presented in Figure \ref{fig:exa06}.
\begin{figure*}[!h]
	\centering
	\begin{subfigure}[t]{0.4\textwidth}
		\centering
		\includegraphics[width=\linewidth]{06/flux1.png} 
		\caption{$\phi_{1}$.}
		\label{fig:exa06-flux1}
	\end{subfigure}
	\vspace{1cm}
	\begin{subfigure}[t]{0.4\textwidth}
		\centering
		\includegraphics[width=\linewidth]{06/flux2.png} 
		\caption{$\phi_{2}$.}
		\label{fig:exa06-flux2}
	\end{subfigure}
	\hfill
	\caption{Flux.}
	\label{fig:exa06}
\end{figure*}

\newpage
\section{Moltres - Example 01}

Kernels involved:
\begin{description}[font=$\bullet$\scshape\bfseries]
	\item[] GroupDiffusion
	\item[] SigmaR
	\item[] InScatter
	\item[] NtSource
\end{description}

Equations involved:
\begin{equation}
-\nabla.(D_{1}\nabla\phi_{1})+\Sigma^{R}_{1}\phi_{1}-S=0
\end{equation}
\begin{equation}
-\nabla.(D_{2}\nabla\phi_{2})-\Sigma^{S}_{1-2}\phi_{1}=0
\end{equation}

Boundary conditions:
\begin{description}[]
	\item[] $\phi_{1}('left')=0$ and $\phi_{1}('right')=0$
	\item[] $\phi_{2}('left')=0$ and $\phi_{2}('right')=0$
\end{description}

The results are presented in Figure \ref{fig:m01}.
\begin{figure*}[!h]
	\centering
	\begin{subfigure}[t]{0.4\textwidth}
		\centering
		\includegraphics[width=\linewidth]{moltres01/flux1.png} 
		\caption{$\phi_{1}$.}
		\label{fig:m01-flux1}
	\end{subfigure}
	\vspace{1cm}
	\begin{subfigure}[t]{0.4\textwidth}
		\centering
		\includegraphics[width=\linewidth]{moltres01/flux2.png} 
		\caption{$\phi_{2}$.}
		\label{fig:m01-flux2}
	\end{subfigure}
	\hfill
	\caption{Flux.}
	\label{fig:m01}
\end{figure*}

\newpage
\section{Moltres - Example 02}

Kernels involved:
\begin{description}[font=$\bullet$\scshape\bfseries]
	\item[] TempDiffusion
	\item[]	ConservativeTemperatureAdvection
	\item[] TemperatureOutflowBC (it is defined by the 2nd BC)
\end{description}

Equations involved:
\begin{equation}
\rho c_{p}\vec{v}\nabla T-\nabla.(k\nabla T)=0
\end{equation}

Boundary conditions:
\begin{description}[]
	\item[] $T('bottom')=0$, $T('left')=1$, and $T('right')=1$
	\item[] $\check{n}\cdot\vec{\Gamma}=\rho c_{p}(\check{n}\cdot\vec{v})T$ if $\check{n}\cdot\vec{v}>0$
	\item[] $\check{n}\cdot\vec{\Gamma}=0$ if $\check{n}\cdot\vec{v}<=0$
\end{description}

The results are presented in Figure \ref{fig:m02}.
\begin{figure*}[!h]
	\centering
	\includegraphics[width=\linewidth]{moltres02/temperature.png} 
	\hfill
	\caption{Temperature.}
	\label{fig:m02}
\end{figure*}



%\newpage
%\section{Example 04}
%
%This example can be found in:
%
%$https://github.com/robfairh/mymoltres/tree/new_kernels/examples/04/$\\
%
%Kernels involved:
%\begin{description}[font=$\bullet$\scshape\bfseries]
%	\item[] MatINSTemperatureTimeDerivative
%	\item[] TempDiffusion
%\end{description}
%
%This case calculates the temperature ($T$).
%
%Equations involved:
%\begin{equation}
%\rho c_{p}\frac{\partial T}{\partial t}-\nabla.(k\nabla T)=0
%\end{equation}
%
%Boundary conditions:
%\begin{description}[]
%	\item[] $T('left')=0$ and $T('right')=1$
%\end{description}
%
%The user defines $\rho=1$, $c_{p}=1$, and $k=1$ for the kernels manually.
%The results are presented in Figure \ref{fig:exa04}.
%
%\begin{figure*}[!h]
%	\centering
%	\begin{subfigure}[t]{0.4\textwidth}
%		\centering
%		\includegraphics[width=\linewidth]{04/temperature-t=0.png} 
%		\caption{$T$ at $t=0$.}
%		\label{fig:exa04-temp0}
%	\end{subfigure}
%	\vspace{1cm}
%	\begin{subfigure}[t]{0.4\textwidth}
%		\centering
%		\includegraphics[width=\linewidth]{04/temperature-t=50.png} 
%		\caption{$T$ at $t=50$.}
%		\label{fig:exa04-temp50}
%	\end{subfigure}
%	\hfill
%	\caption{Temperature.}
%	\label{fig:exa04}
%\end{figure*}
%
%
%\newpage
%\section{Example 06}
%
%This example can be found in:
%
%$https://github.com/robfairh/mymoltres/tree/new_kernels/examples/06/$\\
%
%Kernels involved:
%\begin{description}[font=$\bullet$\scshape\bfseries]
%	\item[] MatINSTemperatureTimeDerivative
%	\item[] TempDiffusion
%\end{description}
%
%This case calculates the temperature ($T$).
%
%Equations involved:
%\begin{equation}
%\rho c_{p}\frac{\partial T}{\partial t}-\nabla.(k\nabla T)=0
%\end{equation}
%
%Boundary conditions:
%\begin{description}[]
%	\item[] $T('left')=0$ and $T('right')=1$
%\end{description}
%
%The user defines $\rho=1$, $c_{p}=1$, and $k=1$ for the kernels manually.
%The results are presented in Figure \ref{fig:exa06}.
%
%\begin{figure*}[!h]
%	\centering
%	\begin{subfigure}[t]{0.4\textwidth}
%		\centering
%		\includegraphics[width=\linewidth]{04/temperature-t=0.png} 
%		\caption{$T$ at $t=0$.}
%		\label{fig:exa06-temp0}
%	\end{subfigure}
%	\vspace{1cm}
%	\begin{subfigure}[t]{0.4\textwidth}
%		\centering
%		\includegraphics[width=\linewidth]{04/temperature-t=50.png} 
%		\caption{$T$ at $t=50$.}
%		\label{fig:exa06-temp50}
%	\end{subfigure}
%	\hfill
%	\caption{Temperature.}
%	\label{fig:exa06}
%\end{figure*}

\pagebreak 
\bibliographystyle{plain}
\bibliography{bibliography}

\end{document}