\documentclass[11pt,letterpaper]{article}
\usepackage[utf8]{inputenc}
\usepackage{caption} % for table captions
\usepackage{amsmath} % for multi-line equations and piecewises
\DeclareMathOperator{\sign}{sign}
\usepackage{graphicx}
\usepackage{relsize}
\usepackage{xspace}
\usepackage{verbatim} % for block comments
\usepackage{subcaption} % for subfigures
\usepackage{enumitem} % for a) b) c) lists
\newcommand{\Cyclus}{\textsc{Cyclus}\xspace}%
\newcommand{\Cycamore}{\textsc{Cycamore}\xspace}%
\newcommand{\deploy}{\texttt{d3ploy}\xspace}%
\newcommand{\Deploy}{\texttt{D3ploy}\xspace}%
\usepackage{tabularx}
\usepackage{color}
\usepackage{multirow}
\usepackage[acronym,toc]{glossaries}
\newacronym[longplural={metric tons of heavy metal}]{MTHM}{MTHM}{metric ton of heavy metal}
\newacronym{ABM}{ABM}{agent-based modeling}
\newacronym{ACDIS}{ACDIS}{Program in Arms Control \& Domestic and International Security}
\newacronym{AHTR}{AHTR}{Advanced High Temperature Reactor}
\newacronym{ANDRA}{ANDRA}{Agence Nationale pour la gestion des D\'echets RAdioactifs, the French National Agency for Radioactive Waste Management}
\newacronym{ANL}{ANL}{Argonne National Laboratory}
\newacronym{API}{API}{application programming interface}
\newacronym{ARCH}{ARCH}{autoregressive conditional heteroskedastic}
\newacronym{ARE}{ARE}{Aircraft Reactor Experiment}
\newacronym{ARFC}{ARFC}{Advanced Reactors and Fuel Cycles}
\newacronym{ARMA}{ARMA}{autoregressive moving average}
\newacronym{ASME}{ASME}{American Society of Mechanical Engineers}
\newacronym{ATWS}{ATWS}{Anticipated Transient Without Scram}
\newacronym{BDBE}{BDBE}{Beyond Design Basis Event}
\newacronym{BIDS}{BIDS}{Berkeley Institute for Data Science}
\newacronym{BOL}{BOL}{Beginning-of-Life}
\newacronym{BSD}{BSD}{Berkeley Software Distribution}
\newacronym{CAFCA}{CAFCA}{ Code for Advanced Fuel Cycles Assessment }
\newacronym{CASL}{CASL}{Consortium for Advanced Simulation of Light Water Reactors}
\newacronym{CDTN}{CDTN}{Centro de Desenvolvimento da Tecnologia Nuclear}
\newacronym{CEA}{CEA}{Commissariat \`a l'\'Energie Atomique et aux \'Energies Alternatives}
\newacronym{CI}{CI}{continuous integration}
\newacronym{CNEC}{CNEC}{Consortium for Nonproliferation Enabling Capabilities}
\newacronym{CNEN}{CNEN}{Comiss\~{a}o Nacional de Energia Nuclear}
\newacronym{CNERG}{CNERG}{Computational Nuclear Engineering Research Group}
\newacronym{COSI}{COSI}{Commelini-Sicard}
\newacronym{COTS}{COTS}{commercial, off-the-shelf}
\newacronym{CSNF}{CSNF}{commercial spent nuclear fuel}
\newacronym{CTAH}{CTAHs}{Coiled Tube Air Heaters}
\newacronym{CUBIT}{CUBIT}{CUBIT Geometry and Mesh Generation Toolkit}
\newacronym{CURIE}{CURIE}{Centralized Used Fuel Resource for Information Exchange}
\newacronym{DAG}{DAG}{directed acyclic graph}
\newacronym{DANESS}{DANESS}{Dynamic Analysis of Nuclear Energy System Strategies}
\newacronym{DBE}{DBE}{Design Basis Event}
\newacronym{DESAE}{DESAE}{Dynamic Analysis of Nuclear Energy Systems Strategies}
\newacronym{DHS}{DHS}{Department of Homeland Security}
\newacronym{DOE}{DOE}{Department of Energy}
\newacronym{DRACS}{DRACS}{Direct Reactor Auxiliary Cooling System}
\newacronym{DRE}{DRE}{dynamic resource exchange}
\newacronym{DSNF}{DSNF}{DOE spent nuclear fuel}
\newacronym{DYMOND}{DYMOND}{Dynamic Model of Nuclear Development }
\newacronym{EBS}{EBS}{Engineered Barrier System}
\newacronym{EDZ}{EDZ}{Excavation Disturbed Zone}
\newacronym{EIA}{EIA}{U.S. Energy Information Administration}
\newacronym{EPA}{EPA}{Environmental Protection Agency}
\newacronym{EP}{EP}{Engineering Physics}
\newacronym{FCO}{FCO}{Fuel Cycle Options}
\newacronym{FCT}{FCT}{Fuel Cycle Technology}
\newacronym{FCWMD}{FCWMD}{Fuel Cycle and Waste Management Division}
\newacronym{FEHM}{FEHM}{Finite Element Heat and Mass Transfer}
\newacronym{FEPs}{FEPs}{Features, Events, and Processes}
\newacronym{FHR}{FHR}{Fluoride-Salt-Cooled High-Temperature Reactor}
\newacronym{FLiBe}{FLiBe}{Fluoride-Lithium-Beryllium}
\newacronym{GCAM}{GCAM}{Global Change Assessment Model}
\newacronym{GDSE}{GDSE}{Generic Disposal System Environment}
\newacronym{GDSM}{GDSM}{Generic Disposal System Model}
\newacronym{GENIUSv1}{GENIUSv1}{Global Evaluation of Nuclear Infrastructure Utilization Scenarios, Version 1}
\newacronym{GENIUSv2}{GENIUSv2}{Global Evaluation of Nuclear Infrastructure Utilization Scenarios, Version 2}
\newacronym{GENIUS}{GENIUS}{Global Evaluation of Nuclear Infrastructure Utilization Scenarios}
\newacronym{GPAM}{GPAM}{Generic Performance Assessment Model}
\newacronym{GRSAC}{GRSAC}{Graphite Reactor Severe Accident Code}
\newacronym{GUI}{GUI}{graphical user interface}
\newacronym{HLW}{HLW}{high level waste}
\newacronym{HPC}{HPC}{high-performance computing}
\newacronym{HTC}{HTC}{high-throughput computing}
\newacronym{HTGR}{HTGR}{High Temperature Gas-Cooled Reactor}
\newacronym{IAEA}{IAEA}{International Atomic Energy Agency}
\newacronym{IEMA}{IEMA}{Illinois Emergency Mangament Agency}
\newacronym{INL}{INL}{Idaho National Laboratory}
\newacronym{IPRR1}{IRP-R1}{Instituto de Pesquisas Radioativas Reator 1}
\newacronym{IRP}{IRP}{Integrated Research Project}
\newacronym{ISFSI}{ISFSI}{Independent Spent Fuel Storage Installation}
\newacronym{ISRG}{ISRG}{Independent Student Research Group}
\newacronym{JFNK}{JFNK}{Jacobian-Free Newton Krylov}
\newacronym{LANL}{LANL}{Los Alamos National Laboratory}
\newacronym{LBNL}{LBNL}{Lawrence Berkeley National Laboratory}
\newacronym{LCOE}{LCOE}{levelized cost of electricity}
\newacronym{LDRD}{LDRD}{laboratory directed research and development}
\newacronym{LFR}{LFR}{Lead-Cooled Fast Reactor}
\newacronym{LGPL}{LGPL}{Lesser GNU Public License}
\newacronym{LLNL}{LLNL}{Lawrence Livermore National Laboratory}
\newacronym{LMFBR}{LMFBR}{Liquid-Metal-cooled Fast Breeder Reactor}
\newacronym{LOFC}{LOFC}{Loss of Forced Cooling}
\newacronym{LOHS}{LOHS}{Loss of Heat Sink}
\newacronym{LOLA}{LOLA}{Loss of Large Area}
\newacronym{LP}{LP}{linear program}
\newacronym{LWR}{LWR}{Light Water Reactor}
\newacronym{MARKAL}{MARKAL}{MARKet and ALlocation}
\newacronym{MA}{MA}{minor actinide}
\newacronym{MCNP}{MCNP}{Monte Carlo N-Particle code}
\newacronym{MILP}{MILP}{mixed-integer linear program}
\newacronym{MIT}{MIT}{the Massachusetts Institute of Technology}
\newacronym{MOAB}{MOAB}{Mesh-Oriented datABase}
\newacronym{MOOSE}{MOOSE}{Multiphysics Object-Oriented Simulation Environment}
\newacronym{MOX}{MOX}{mixed oxide}
\newacronym{MSBR}{MSBR}{Molten Salt Breeder Reactor}
\newacronym{MSRE}{MSRE}{Molten Salt Reactor Experiment}
\newacronym{MSR}{MSR}{Molten Salt Reactor}
\newacronym{NAGRA}{NAGRA}{National Cooperative for the Disposal of Radioactive Waste}
\newacronym{NCSA}{NCSA}{National Center for Supercomputing Applications}
\newacronym{NEAMS}{NEAMS}{Nuclear Engineering Advanced Modeling and Simulation}
\newacronym{NEUP}{NEUP}{Nuclear Energy University Programs}
\newacronym{NFCSim}{NFCSim}{Nuclear Fuel Cycle Simulator}
\newacronym{NFC}{NFC}{Nuclear Fuel Cycle}
\newacronym{NGNP}{NGNP}{Next Generation Nuclear Plant}
\newacronym{NMWPC}{NMWPC}{Nuclear MW Per Capita}
\newacronym{NNSA}{NNSA}{National Nuclear Security Administration}
\newacronym{NPRE}{NPRE}{Department of Nuclear, Plasma, and Radiological Engineering}
\newacronym{NQA1}{NQA-1}{Nuclear Quality Assurance - 1}
\newacronym{NRC}{NRC}{Nuclear Regulatory Commission}
\newacronym{NSF}{NSF}{National Science Foundation}
\newacronym{NSSC}{NSSC}{Nuclear Science and Security Consortium}
\newacronym{NUWASTE}{NUWASTE}{Nuclear Waste Assessment System for Technical Evaluation}
\newacronym{NWF}{NWF}{Nuclear Waste Fund}
\newacronym{NWTRB}{NWTRB}{Nuclear Waste Technical Review Board}
\newacronym{OCRWM}{OCRWM}{Office of Civilian Radioactive Waste Management}
\newacronym{ORION}{ORION}{ORION}
\newacronym{ORNL}{ORNL}{Oak Ridge National Laboratory}
\newacronym{PARCS}{PARCS}{Purdue Advanced Reactor Core Simulator}
\newacronym{PBAHTR}{PB-AHTR}{Pebble Bed Advanced High Temperature Reactor}
\newacronym{PBFHR}{PB-FHR}{Pebble-Bed Fluoride-Salt-Cooled High-Temperature Reactor}
\newacronym{PEI}{PEI}{Peak Environmental Impact}
\newacronym{PH}{PRONGHORN}{PRONGHORN}
\newacronym{PI}{PI}{Principal Investigator}
\newacronym{PNNL}{PNNL}{Pacific Northwest National Laboratory}
\newacronym{PRIS}{PRIS}{Power Reactor Information System}
\newacronym{PRKE}{PRKE}{Point Reactor Kinetics Equations}
\newacronym{PSPG}{PSPG}{Pressure-Stabilizing/Petrov-Galerkin}
\newacronym{PWAR}{PWAR}{Pratt and Whitney Aircraft Reactor}
\newacronym{PWR}{PWR}{Pressurized Water Reactor}
\newacronym{PyNE}{PyNE}{Python toolkit for Nuclear Engineering}
\newacronym{PyRK}{PyRK}{Python for Reactor Kinetics}
\newacronym{QA}{QA}{quality assurance}
\newacronym{RDD}{RD\&D}{Research Development and Demonstration}
\newacronym{RD}{R\&D}{Research and Development}
\newacronym{RELAP}{RELAP}{Reactor Excursion and Leak Analysis Program}
\newacronym{RIA}{RIA}{Reactivity Insertion Accident}
\newacronym{RIF}{RIF}{Region-Institution-Facility}
\newacronym{SAM}{SAM}{Simulation and Modeling}
\newacronym{SCF}{SCF}{Software Carpentry Foundation}
\newacronym{SFR}{SFR}{Sodium-Cooled Fast Reactor}
\newacronym{SINDAG}{SINDA{\textbackslash}G}{Systems Improved Numerical Differencing Analyzer $\backslash$ Gaski}
\newacronym{SKB}{SKB}{Svensk K\"{a}rnbr\"{a}nslehantering AB}
\newacronym{SNF}{SNF}{spent nuclear fuel}
\newacronym{SNL}{SNL}{Sandia National Laboratory}
\newacronym{SNM}{SNM}{Special Nuclear Material}
\newacronym{STC}{STC}{specific temperature change}
\newacronym{SUPG}{SUPG}{Streamline-Upwind/Petrov-Galerkin}
\newacronym{SWF}{SWF}{Separations and Waste Forms}
\newacronym{SWU}{SWU}{Separative Work Unit}
\newacronym{SandO}{S\&O}{Signatures and Observables}
\newacronym{THW}{THW}{The Hacker Within}
\newacronym{TRIGA}{TRIGA}{Training Research Isotope General Atomic}
\newacronym{TRISO}{TRISO}{Tristructural Isotropic}
\newacronym{TSM}{TSM}{Total System Model}
\newacronym{TSPA}{TSPA}{Total System Performance Assessment for the Yucca Mountain License Application}
\newacronym{UDB}{UDB}{Unified Database}
\newacronym{UFD}{UFD}{Used Fuel Disposition}
\newacronym{UML}{UML}{Unified Modeling Language}
\newacronym{UNFSTANDARDS}{UNFST\&DARDS}{Used Nuclear Fuel Storage, Transportation \& Disposal Analysis Resource and Data System}
\newacronym{UOX}{UOX}{uranium oxide}
\newacronym{UQ}{UQ}{uncertainty quantification}
\newacronym{US}{US}{United States}
\newacronym{UW}{UW}{University of Wisconsin}
\newacronym{VISION}{VISION}{the Verifiable Fuel Cycle Simulation Model}
\newacronym{VV}{V\&V}{verification and validation}
\newacronym{WIPP}{WIPP}{Waste Isolation Pilot Plant}
\newacronym{YMG}{YMG}{Young Members Group}
\newacronym{YMR}{YMR}{Yucca Mountain Repository Site}
\newacronym{NEI}{NEI}{Nuclear Energy Institute}
%\newacronym{<++>}{<++>}{<++>}
%\newacronym{<++>}{<++>}{<++>}

\definecolor{bg}{rgb}{0.95,0.95,0.95}
\newcolumntype{b}{X}
\newcolumntype{f}{>{\hsize=.15\hsize}X}
\newcolumntype{s}{>{\hsize=.5\hsize}X}
\newcolumntype{m}{>{\hsize=.75\hsize}X}
\newcolumntype{r}{>{\hsize=1.1\hsize}X}
\usepackage{titling}
\usepackage[hang,flushmargin]{footmisc}
\renewcommand*\footnoterule{}
\usepackage{tikz}

\usetikzlibrary{shapes.geometric,arrows}
\tikzstyle{process} = [rectangle, rounded corners, 
minimum width=1cm, minimum height=1cm,text centered, draw=black, 
fill=blue!30]
\tikzstyle{arrow} = [thick,->,>=stealth]

\graphicspath{}
\title{An Introduction to Moltres}
\author{Roberto E. Fairhurst Agosta}

\begin{document}
	\begin{titlepage}
		\maketitle
		\thispagestyle{empty}
	\end{titlepage}
	
\section{Example 00}

Kernels involved:
\begin{description}[font=$\bullet$\scshape\bfseries]
	\item[] moose/Diffusion
\end{description}

Equations involved:
\begin{equation}
-\nabla.\nabla T=0
\end{equation}

Boundary conditions:
\begin{description}[]
	\item[] $T('left')=0$ and $T('right')=1$
\end{description}

The results are presented in Figure \ref{fig:exa00}.
\begin{figure*}[!h]
	\centering
	\includegraphics[width=\linewidth]{00/rectangularmesh.png} 
	\hfill
	\caption{Temperature.}
	\label{fig:exa00}
\end{figure*}

\newpage
\section{Example 01}

Kernels involved:
\begin{description}[font=$\bullet$\scshape\bfseries]
	\item[] moose/Diffusion
	\item[] Convection
\end{description}

Equations involved:
\begin{equation}
-\nabla.\nabla T + \vec{v}\nabla T=0
\end{equation}

Boundary conditions:
\begin{description}[]
	\item[] $T('bottom')=0$, $T('left')=1$, and $T('right')=1$
\end{description}

The results are presented in Figure \ref{fig:exa01}.
\begin{figure*}[!h]
	\centering
	\includegraphics[width=\linewidth]{01/temperature.png} 
	\hfill
	\caption{Temperature.}
	\label{fig:exa01}
\end{figure*}

\newpage
\section{Example 02}

Kernels involved:
\begin{description}[font=$\bullet$\scshape\bfseries]
	\item[] NtDiffusion
	\item[] NtSource
\end{description}

Equations involved:
\begin{equation}
-\nabla.(D\nabla \phi) - S=0
\end{equation}

Boundary conditions:
\begin{description}[]
	\item[] $\phi('left')=0$, and $\phi('right')=0$
\end{description}

The results are presented in Figure \ref{fig:exa02}.
\begin{figure*}[!h]
	\centering
	\includegraphics[width=\linewidth]{02/constantsource.png} 
	\hfill
	\caption{Flux.}
	\label{fig:exa02}
\end{figure*}

\newpage
\section{Example 03}

Kernels involved:
\begin{description}[font=$\bullet$\scshape\bfseries]
	\item[] NtDiffusion
	\item[] NtSigmaA
\end{description}

Equations involved:
\begin{equation}
-\nabla.(D\nabla \phi) + \Sigma_{A}\phi =0
\end{equation}

Boundary conditions:
\begin{description}[]
	\item[] $\phi('left')=0$, and $\phi('right')=1$
\end{description}

The results are presented in Figure \ref{fig:exa03}.
\begin{figure*}[!h]
	\centering
	\includegraphics[width=\linewidth]{03/absorption.png} 
	\hfill
	\caption{Flux.}
	\label{fig:exa03}
\end{figure*}

\newpage
\section{Example 04}

Kernels involved:
\begin{description}[font=$\bullet$\scshape\bfseries]
	\item[] NtDiffusion
	\item[] NtSigmaF
\end{description}

Equations involved:
\begin{equation}
-\nabla.(D\nabla \phi) - \nu\Sigma_{F}\phi =0
\end{equation}

Boundary conditions:
\begin{description}[]
	\item[] $\phi('left')=0$, and $\phi('right')=1$
\end{description}

The results are presented in Figure \ref{fig:exa04}.
\begin{figure*}[!h]
	\centering
	\includegraphics[width=\linewidth]{04/fission.png} 
	\hfill
	\caption{Flux.}
	\label{fig:exa04}
\end{figure*}

\newpage
\section{Example 05}

Kernels involved:
\begin{description}[font=$\bullet$\scshape\bfseries]
	\item[] NtTimeDerivative
	\item[] NtDiffusion
\end{description}

Equations involved:
\begin{equation}
a\frac{\partial\phi}{\partial t}-\nabla.(D\nabla \phi)=0
\end{equation}

Boundary conditions:
\begin{description}[]
	\item[] $\phi('bottom')=0$, and $\phi('top')=1$
\end{description}

The results are presented in Figure \ref{fig:exa05}.
\begin{figure*}[!h]
	\centering
	\begin{subfigure}[t]{0.4\textwidth}
		\centering
		\includegraphics[width=\linewidth]{05/timederivative_t=0.png} 
		\caption{$\phi$ at t=0.}
		\label{fig:exa01-flux1}
	\end{subfigure}
	\vspace{1cm}
	\begin{subfigure}[t]{0.4\textwidth}
		\centering
		\includegraphics[width=\linewidth]{05/timederivative_t=50.png} 
		\caption{$\phi$ at t=50.}
		\label{fig:exa01-flux2}
	\end{subfigure}
	\hfill
	\caption{Flux.}
	\label{fig:exa05}
\end{figure*}

\newpage
\section{Example 06}

Kernels involved:
\begin{description}[font=$\bullet$\scshape\bfseries]
	\item[] NtDiffusion
	\item[] InScatter
	\item[] NtSource
\end{description}

Equations involved:
\begin{equation}
-\nabla.(D_{1}\nabla \phi_{1})-S=0
\end{equation}
\begin{equation}
-\nabla.(D_{2}\nabla \phi_{2})-\Sigma_{S1-2}\phi_{1}=0
\end{equation}

Boundary conditions:
\begin{description}[]
	\item[] $\phi_{1}('left')=0$, and $\phi_{1}('right')=0$
	\item[] $\phi_{2}('left')=0$, and $\phi_{2}('right')=0$
\end{description}

The results are presented in Figure \ref{fig:exa06}.
\begin{figure*}[!h]
	\centering
	\begin{subfigure}[t]{0.4\textwidth}
		\centering
		\includegraphics[width=\linewidth]{06/flux1.png} 
		\caption{$\phi_{1}$.}
		\label{fig:exa06-flux1}
	\end{subfigure}
	\vspace{1cm}
	\begin{subfigure}[t]{0.4\textwidth}
		\centering
		\includegraphics[width=\linewidth]{06/flux2.png} 
		\caption{$\phi_{2}$.}
		\label{fig:exa06-flux2}
	\end{subfigure}
	\hfill
	\caption{Flux.}
	\label{fig:exa06}
\end{figure*}

\newpage
\section{Moltres - Example 01}

Kernels involved:
\begin{description}[font=$\bullet$\scshape\bfseries]
	\item[] GroupDiffusion
	\item[] SigmaR
	\item[] InScatter
	\item[] NtSource
\end{description}

Equations involved:
\begin{equation}
-\nabla.(D_{1}\nabla\phi_{1})+\Sigma^{R}_{1}\phi_{1}-S=0
\end{equation}
\begin{equation}
-\nabla.(D_{2}\nabla\phi_{2})-\Sigma^{S}_{1-2}\phi_{1}=0
\end{equation}

Boundary conditions:
\begin{description}[]
	\item[] $\phi_{1}('left')=0$ and $\phi_{1}('right')=0$
	\item[] $\phi_{2}('left')=0$ and $\phi_{2}('right')=0$
\end{description}

The results are presented in Figure \ref{fig:m01}.
\begin{figure*}[!h]
	\centering
	\begin{subfigure}[t]{0.4\textwidth}
		\centering
		\includegraphics[width=\linewidth]{moltres01/flux1.png} 
		\caption{$\phi_{1}$.}
		\label{fig:m01-flux1}
	\end{subfigure}
	\vspace{1cm}
	\begin{subfigure}[t]{0.4\textwidth}
		\centering
		\includegraphics[width=\linewidth]{moltres01/flux2.png} 
		\caption{$\phi_{2}$.}
		\label{fig:m01-flux2}
	\end{subfigure}
	\hfill
	\caption{Flux.}
	\label{fig:m01}
\end{figure*}

\newpage
\section{Moltres - Example 02}

Kernels involved:
\begin{description}[font=$\bullet$\scshape\bfseries]
	\item[] TempDiffusion
	\item[]	ConservativeTemperatureAdvection
	\item[] TemperatureOutflowBC (it is defined by the 2nd BC)
\end{description}

Equations involved:
\begin{equation}
\rho c_{p}\vec{v}\nabla T-\nabla.(k\nabla T)=0
\end{equation}

Boundary conditions:
\begin{description}[]
	\item[] $T('bottom')=0$, $T('left')=1$, and $T('right')=1$
	\item[] $\check{n}\cdot\vec{\Gamma}=\rho c_{p}(\check{n}\cdot\vec{v})T$ if $\check{n}\cdot\vec{v}>0$
	\item[] $\check{n}\cdot\vec{\Gamma}=0$ if $\check{n}\cdot\vec{v}<=0$
\end{description}

The results are presented in Figure \ref{fig:m02}.
\begin{figure*}[!h]
	\centering
	\includegraphics[width=\linewidth]{moltres02/temperature.png} 
	\hfill
	\caption{Temperature.}
	\label{fig:m02}
\end{figure*}

\newpage
\section{Example 07}

Kernels involved:
\begin{description}[font=$\bullet$\scshape\bfseries]
	\item[] TempDiffusion
	\item[] TempSource
\end{description}

Equations involved:
\begin{equation}
-\nabla.(D\nabla T)-S=0
\end{equation}

Boundary conditions:
\begin{description}[]
	\item[] $T('left')=0$, and $T('right')=0$
\end{description}

\begin{table}[]
	\centering
	\caption{Different values of K.}
	\begin{tabular}{|l|l|l|}
		\hline
		x  & \multicolumn{1}{c|}{yA} & yB \\ \hline
		0  & 1                       & 1  \\ \hline
		20 & 1                       & 10 \\ \hline
	\end{tabular}
	\label{tab:ex07}
\end{table}

The results are presented in Figure \ref{fig:exa07}.
\begin{figure*}[!h]
	\centering
	\begin{subfigure}[t]{0.4\textwidth}
		\centering
		\includegraphics[width=\linewidth]{07/ex07A.png} 
		\caption{$T$ K in the list A.}
		\label{fig:exa07-temp1}
	\end{subfigure}
	\vspace{1cm}
	\begin{subfigure}[t]{0.4\textwidth}
		\centering
		\includegraphics[width=\linewidth]{07/ex07B.png} 
		\caption{$T$ K in the list B.}
		\label{fig:exa07-temp2}
	\end{subfigure}
	\hfill
	\caption{Temperature.}
	\label{fig:exa07}
\end{figure*}

\newpage
\section{Moltres - Example 03}

Kernels involved:
\begin{description}[font=$\bullet$\scshape\bfseries]
	\item[] moose/MatDiffusion
	\item[]	ConservativeTemperatureAdvection
	\item[] TempSource
	\item[] TemperatureOutflowBC (it is defined by the 2nd BC)
\end{description}

Equations involved:
\begin{equation}
\rho c_{p}\vec{v}\nabla T-\nabla.(k\nabla T)-S=0
\end{equation}

Boundary conditions:
\begin{description}[]
	\item[] $T('bottom')=0$, $T('left')=0$, and $T('right')=0$
	\item[] $\check{n}\cdot\vec{\Gamma}=\rho c_{p}(\check{n}\cdot\vec{v})T$ if $\check{n}\cdot\vec{v}>0$
	\item[] $\check{n}\cdot\vec{\Gamma}=0$ if $\check{n}\cdot\vec{v}<=0$
\end{description}

The results are presented in Figure \ref{fig:m03}.
\begin{figure*}[!h]
	\centering
	\includegraphics[width=\linewidth]{moltres03/k_cp_rho=1.png} 
	\hfill
	\caption{Temperature.}
	\label{fig:m03}
\end{figure*}

\newpage
\section{Moltres - Example 04}

This example uses \textit{GenericMoltresMaterial} to the define the properties of the fuel.

Kernels involved:
\begin{description}[font=$\bullet$\scshape\bfseries]
	\item[] GroupDiffusion
	\item[]	SigmaR
	\item[] InScatter
	\item[] NtSource
\end{description}

Equations involved:
\begin{equation}
-\nabla.(D_{1}\nabla\phi_{1})+\Sigma^{R}_{1}\phi_{1}-S=0
\end{equation}
\begin{equation}
-\nabla.(D_{2}\nabla\phi_{2})-\Sigma^{S}_{1-2}\phi_{1}=0
\end{equation}

Boundary conditions:
\begin{description}[]
	\item[] $\phi_{1}('left')=0$ and $\phi_{1}('right')=0$
	\item[] $\phi_{2}('left')=0$ and $\phi_{2}('right')=0$
\end{description}

The results are presented in Figure \ref{fig:m04}.
\begin{figure*}[!h]
	\centering
	\begin{subfigure}[t]{0.4\textwidth}
		\centering
		\includegraphics[width=\linewidth]{moltres04/flux1.png} 
		\caption{$\phi_{1}$.}
		\label{fig:m04-flux1}
	\end{subfigure}
	\vspace{1cm}
	\begin{subfigure}[t]{0.4\textwidth}
		\centering
		\includegraphics[width=\linewidth]{moltres04/flux2.png} 
		\caption{$\phi_{2}$.}
		\label{fig:m04-flux2}
	\end{subfigure}
	\hfill
	\caption{Flux.}
	\label{fig:m04}
\end{figure*}

\newpage
\section{Moltres - Example 05}

This example uses \textit{GenericMoltresMaterial} to the define the properties of the fuel.

Kernels involved:
\begin{description}[font=$\bullet$\scshape\bfseries]
    \item[] NtTimeDerivative
	\item[] GroupDiffusion
	\item[]	SigmaR
	\item[] InScatter
	\item[] CoupledFissionKernel
	\item[] NtSource
	\item[] MatINSTemperatureTimeDerivative
	\item[]	MatDiffusion
	\item[] ConservativeTemperatureAdvection
	\item[] TransientFissionHeatSource
    \item[]	TemperatureOutflowBC 
\end{description}

Equations involved:
\begin{equation}
\frac{1}{v_{1}}\frac{\partial \phi_{1}}{\partial t}-\nabla.(D_{1}\nabla\phi_{1})+\Sigma^{R}_{1}\phi_{1}-\Sigma^{S}_{2-1}\phi_{2}-\chi_{1}\sum_{g}\nu\Sigma^{F}_{g}\phi_{g}-S=0
\end{equation}
\begin{equation}
\frac{1}{v_{2}}\frac{\partial \phi_{2}}{\partial t}-\nabla.(D_{2}\nabla\phi_{2})+\Sigma^{R}_{2}\phi_{2}-\Sigma^{S}_{1-2}\phi_{1}-\chi_{2}\sum_{g}\nu\Sigma^{F}_{g}\phi_{g}=0
\end{equation}
\begin{equation}
-\nabla.(k\nabla T)+\rho c_{p} (\frac{\partial T}{\partial t} + \vec{v} \nabla T)-\sum_{g}\epsilon_{g}\Sigma^{F}_{g}\phi_{g}=0
\end{equation}

Boundary conditions:
\begin{description}[]
	\item[] $\phi_{1}('left')=0$ and $\phi_{1}('right')=0$
	\item[] $\phi_{2}('left')=0$ and $\phi_{2}('right')=0$
	\item[] $T('bottom')=0$, $T('left')=0$, and $T('right')=0$
	\item[] $\check{n}\cdot\vec{\Gamma}=\rho c_{p}(\check{n}\cdot\vec{v})T$ if $\check{n}\cdot\vec{v}>0$
	\item[] $\check{n}\cdot\vec{\Gamma}=0$ if $\check{n}\cdot\vec{v}<=0$
\end{description}

The results are presented in Figures \ref{fig:m05A}, \ref{fig:m05B}, and \ref{fig:m05C}.
\begin{figure*}[!h]
	\centering
	\begin{subfigure}[t]{0.4\textwidth}
		\centering
		\includegraphics[width=\linewidth]{moltres05/flux1-t=0.png} 
		\caption{$\phi_{1}$ at t=0.}
		\label{fig:m05-flux1-1}
	\end{subfigure}
	\vspace{1cm}
	\begin{subfigure}[t]{0.4\textwidth}
		\centering
		\includegraphics[width=\linewidth]{moltres05/flux1-t=4.png} 
		\caption{$\phi_{1}$ at t=4.}
		\label{fig:m05-flux1-5}
	\end{subfigure}
	\hfill
	\caption{Flux1.}
	\label{fig:m05A}
\end{figure*}

\begin{figure*}[!h]
	\centering
	\begin{subfigure}[t]{0.4\textwidth}
		\centering
		\includegraphics[width=\linewidth]{moltres05/flux2-t=0.png} 
		\caption{$\phi_{2}$ at t=0.}
		\label{fig:m05-flux2-1}
	\end{subfigure}
	\vspace{1cm}
	\begin{subfigure}[t]{0.4\textwidth}
		\centering
		\includegraphics[width=\linewidth]{moltres05/flux2-t=4.png} 
		\caption{$\phi_{2}$ at t=4.}
		\label{fig:m05-flux2-5}
	\end{subfigure}
	\hfill
	\caption{Flux2.}
	\label{fig:m05B}
\end{figure*}

\begin{figure*}[!h]
	\centering
	\begin{subfigure}[t]{0.4\textwidth}
		\centering
		\includegraphics[width=\linewidth]{moltres05/temp-t=0.png} 
		\caption{$T$ at t=0.}
		\label{fig:m05-temp-1}
	\end{subfigure}
	\vspace{1cm}
	\begin{subfigure}[t]{0.4\textwidth}
		\centering
		\includegraphics[width=\linewidth]{moltres05/temp-t=4.png} 
		\caption{$T$ at t=4.}
		\label{fig:m05-temp-5}
	\end{subfigure}
	\hfill
	\caption{Temperature.}
	\label{fig:m05C}
\end{figure*}

\newpage
\section{Example 08}

This example defines two blocks in the mesh: \textbf{fuel} and \textbf{reflector}.

Kernels involved:
\begin{description}[font=$\bullet$\scshape\bfseries]
	\item[] TempDiffusion
	\item[] TempSource
\end{description}

Equations involved:
\begin{equation}
-\nabla.(D\nabla T)-S=0
\end{equation}

Boundary conditions:
\begin{description}[]
	\item[] $T('left')=0$, and $T('right')=0$
\end{description}

The results are presented in Figure \ref{fig:exa08}.
\begin{figure*}[!h]
	\centering
	\includegraphics[width=\linewidth]{08/temp.png}
	\hfill
	\caption{Temperature.}
	\label{fig:exa08}
\end{figure*}

\section{N-S: Example 01}

Kernels involved:
\begin{description}[font=$\bullet$\scshape\bfseries]
	\item[] moose/navier\_stokes/INSMass
	\item[] moose/navier\_stokes/INSMomentumLaplaceForm
\end{description}

Equations involved:
\begin{equation}
\rho (\vec{v} . \nabla)\vec{v}+\nabla p - \mu \nabla^{2}\vec{v}=0
\end{equation}

Boundary conditions:
\begin{description}[]
	\item[] $v_{x}('top')=0$ and $v_{x}('bottom')=0$
	\item[] $v_{y}('left')=0, v_{y}('top')=0$, and $v_{y}('bottom')=0$
	\item[] $p('left')=1$ and $p('right')=0$
\end{description}

The results are presented in Figure \ref{fig:ns01}.
\begin{figure*}[!h]
	\centering
	\includegraphics[width=\linewidth]{ns01/velocity.png} 
	\hfill
	\caption{Velocity.}
	\label{fig:ns01}
\end{figure*}

\section{N-S: Example 02}

Kernels involved:
\begin{description}[font=$\bullet$\scshape\bfseries]
	\item[] moose/navier\_stokes/INSMass
	\item[] moose/navier\_stokes/INSMomentumLaplaceForm
	\item[] moose/navier\_stokes/INSMomentumTimeDerivative
\end{description}

Equations involved:
\begin{equation}
\rho (\frac{\partial \vec{v}}{\partial t}+(\vec{v} . \nabla)\vec{v})+\nabla p - \mu \nabla^{2}\vec{v}=0
\end{equation}

Boundary conditions:
\begin{description}[]
	\item[] $v_{x}('top')=0$ and $v_{x}('bottom')=0$
	\item[] $v_{y}('left')=0, v_{y}('top')=0$, and $v_{y}('bottom')=0$
	\item[] $p('left')=1$ and $p('right')=0$
\end{description}

The results are presented in Figure \ref{fig:ns02}.
\begin{figure*}[!h]
	\centering
	\begin{subfigure}[t]{0.4\textwidth}
		\centering
		\includegraphics[width=\linewidth]{ns02/vel-t=0.png} 
		\caption{$|\vec{v}|$ at t=0.}
		\label{fig:ns02-v-1}
	\end{subfigure}
	\vspace{1cm}
	\begin{subfigure}[t]{0.4\textwidth}
		\centering
		\includegraphics[width=\linewidth]{ns02/vel-t=4.png}
		\caption{$|\vec{v}|$ at t=4.}
		\label{fig:ns02-v-5}
	\end{subfigure}
	\hfill
	\caption{Flux2.}
	\label{fig:ns02}
\end{figure*}

\section{N-S: Example 03}

Kernels involved:
\begin{description}[font=$\bullet$\scshape\bfseries]
	\item[] moose/navier\_stokes/INSMass
	\item[] moose/navier\_stokes/INSMomentumLaplaceForm
	\item[] moose/MatDiffusion
	\item[] moose/navier\_stokes/INSTemperature
\end{description}

Equations involved:
\begin{equation}
\rho (\vec{v} . \nabla)\vec{v}+\nabla p - \mu \nabla^{2}\vec{v}=0
\end{equation}
\begin{equation}
-\nabla(k \nabla . T)+\rho c_{p}(\vec{v} . \nabla)T =0
\end{equation}

Boundary conditions:
\begin{description}[]
	\item[] $v_{x}('top')=0$ and $v_{x}('bottom')=0$
	\item[] $v_{y}('left')=0, v_{y}('top')=0$, and $v_{y}('bottom')=0$
	\item[] $p('left')=0.5$ and $p('right')=0$
	\item[] $T('left')=0$, $T('top')=100$, and $T('bottom')=100$
\end{description}

The results are presented in Figure \ref{fig:ns03}.
\begin{figure*}[!h]
	\centering
	\begin{subfigure}[t]{0.4\textwidth}
		\centering
		\includegraphics[width=\linewidth]{ns03/temp.png} 
		\caption{Temperature.}
		\label{fig:ns03-temp}
	\end{subfigure}
	\vspace{1cm}
	\begin{subfigure}[t]{0.4\textwidth}
		\centering
		\includegraphics[width=\linewidth]{ns03/vel.png}
		\caption{Velocity.}
		\label{fig:ns03-vel}
	\end{subfigure}
	\hfill
	\caption{Flux2.}
	\label{fig:ns03}
\end{figure*}

\section{N-S: Example 04}

This example has an error. It uses the MatDiffusion Kernel, and the kernel INSTemperature has both the effects of diffusion and convection.

Kernels involved:
\begin{description}[font=$\bullet$\scshape\bfseries]
	\item[] moose/navier\_stokes/INSMass
	\item[] moose/navier\_stokes/
	INSMomentumTimeDerivative
	\item[] moose/navier\_stokes/INSMomentumLaplaceForm
	\item[] moose/navier\_stokes/MatINSTemperatureTimeDerivative
	\item[] moose/MatDiffusion
	\item[] moose/navier\_stokes/INSTemperature
\end{description}

Equations involved:
\begin{equation}
\rho (\frac{\partial \vec{v}}{\partial t}+\vec{v} . \nabla\vec{v})+\nabla p - \mu \nabla^{2}\vec{v}=0
\end{equation}
\begin{equation}
-\nabla(k \nabla . T)+\rho c_{p}(\frac{\partial T}{\partial t}+\vec{v} . \nabla T)=0
\end{equation}

Boundary conditions:
\begin{description}[]
	\item[] $v_{x}('left')=0$, $v_{x}('top')=0$ and $v_{x}('bottom')=0$
	\item[] $v_{y}('left')=0$, and $v_{y}('right')=0$
	\item[] $p('bottom')=0.5$ and $p('top')=0$
	\item[] $T('bottom')=0$, $q''('left')=1$, and $q''('right')=-1$
	(Here the BC are positive in one side and negative on the other side, the input file defines them as both positive.)
\end{description}

The results are presented in Figures \ref{fig:ns04a} and \ref{fig:ns04b}.
\begin{figure*}[!h]
	\centering
	\begin{subfigure}[t]{0.4\textwidth}
		\centering
		\includegraphics[width=\linewidth]{ns04/temp-t=0.png} 
		\caption{Temperature t=0.}
		\label{fig:ns04-temp-t=0}
	\end{subfigure}
	\vspace{1cm}
	\begin{subfigure}[t]{0.4\textwidth}
		\centering
		\includegraphics[width=\linewidth]{ns04/temp-t=4.png}
		\caption{Temperature t=4.}
		\label{fig:ns04-temp-t=4}
	\end{subfigure}
	\hfill
	\caption{Temperature.}
	\label{fig:ns04a}
\end{figure*}

\begin{figure*}[!h]
	\centering
	\begin{subfigure}[t]{0.4\textwidth}
		\centering
		\includegraphics[width=\linewidth]{ns04/axial-t=0.png} 
		\caption{Axial temperature at t=0.}
		\label{fig:ns04-axial}
	\end{subfigure}
	\vspace{1cm}
	\begin{subfigure}[t]{0.4\textwidth}
		\centering
		\includegraphics[width=\linewidth]{ns04/tempx-t=0.png}
		\caption{Temperature in x-direction at t=0.}
		\label{fig:ns04-tempx}
	\end{subfigure}
	\hfill
	\caption{Temperature.}
	\label{fig:ns04b}
\end{figure*}

\section{N-S: Example 05}

Kernels involved:
\begin{description}[font=$\bullet$\scshape\bfseries]
	\item[] moose/navier\_stokes/INSMass
	\item[] moose/navier\_stokes/
	INSMomentumTimeDerivative
	\item[] moose/navier\_stokes/INSMomentumLaplaceForm
	\item[] INSBoussinesqBodyForce
	\item[] moose/navier\_stokes/MatINSTemperatureTimeDerivative
	\item[] moose/navier\_stokes/INSTemperature
\end{description}

Equations involved:
\begin{equation}
\rho (\frac{\partial \vec{v}}{\partial t}+\vec{v} . \nabla\vec{v})+\nabla p - \mu \nabla^{2}\vec{v} + \alpha g (T - T_{o})\check{j} = 0
\end{equation}
\begin{equation}
-\nabla(k \nabla . T)+\rho c_{p}(\frac{\partial T}{\partial t}+\vec{v} . \nabla T)=0
\end{equation}

Boundary conditions:
\begin{description}[]
	\item[] $v_{x}('left')=0$, $v_{x}('right')=0$, $v_{x}('top')=0$, and $v_{x}('bottom')=0$
	\item[] $v_{y}('left')=0$, $v_{y}('right')=0$, $v_{y}('top')=0$,
	and $v_{y}('bottom')=0$
	\item[] $p(0,0)=0$
	\item[] $T('left')=300$, $T_{o}=200$
\end{description}

The results are presented in Figures \ref{fig:ns05}.
\begin{figure*}[!h]
	\centering
	\begin{subfigure}[t]{0.4\textwidth}
		\centering
		\includegraphics[width=\linewidth]{ns05/ux.png} 
		\caption{ux.}
		\label{fig:ns05-ux}
	\end{subfigure}
	\vspace{1cm}
	\begin{subfigure}[t]{0.4\textwidth}
		\centering
		\includegraphics[width=\linewidth]{ns05/uy.png}
		\caption{uy.}
		\label{fig:ns05-uy}
	\end{subfigure}
	\hfill
	\caption{Temperature.}
	\label{fig:ns05}
\end{figure*}

\section{N-S: Example 06}

This case presents a very simplified version of the moose's INS kernels.

Kernels involved:
\begin{description}[font=$\bullet$\scshape\bfseries]
	\item[] NSMass
	\item[] NSMomentum
\end{description}

Equations involved:
\begin{equation}
\nabla p - \mu \nabla^{2}\vec{v} = 0
\end{equation}
\begin{equation}
\nabla.\vec{v}=0
\end{equation}

Boundary conditions:
\begin{description}[]
	\item[] $v_{x}('top')=0$, and $v_{x}('bottom')=0$
	\item[] $v_{y}('left')=0$, $v_{y}('right')=0$, $v_{y}('bottom')=0$, and $v_{y}('top')=0$
	\item[] $p('left')=1$ and $p('right')=0$
\end{description}

The results are presented in Figure \ref{fig:ns06}.
\begin{figure*}[!h]
	\centering
	\includegraphics[width=\linewidth]{ns06/velocity.png} 
	\hfill
	\caption{Velocity.}
	\label{fig:ns06}
\end{figure*}

\section{N-S: Example 07}

This is still a simplified version of moose's INS kernels, but this time it includes everything for Incompressible N-S.

Kernels involved:
\begin{description}[font=$\bullet$\scshape\bfseries]
	\item[] NSMass
	\item[] NSMomentum
\end{description}

Equations involved:
\begin{equation}
\rho (\vec{v}.\nabla)\vec{v} + \nabla p - \rho g \check{j} - \mu \nabla^{2}\vec{v} = 0
\end{equation}
\begin{equation}
\nabla.\vec{v}=0
\end{equation}

Boundary conditions:
\begin{description}[]
	\item[] $v_{x}('left')=0$, $v_{x}('bottom')=0$, and $v_{x}('top')=0$.
	\item[] $v_{y}('left')=0$ and $v_{y}('right')=0$
	\item[] $p('bottom')=1$ and $p('top')=0$
\end{description}

The results are presented in Figure \ref{fig:ns07}. We can see that the gravity term overcomes the pressure gradient.
\begin{figure*}[!h]
	\centering
	\includegraphics[width=\linewidth]{ns07/vel_y.png} 
	\hfill
	\caption{Velocity.}
	\label{fig:ns07}
\end{figure*}

\section{N-S: Example 08}

This is still a simplified version of moose's INS kernels, but this time it includes everything for Incompressible N-S and also INSTemperature slightly modified.

Kernels involved:
\begin{description}[font=$\bullet$\scshape\bfseries]
	\item[] NSMass
	\item[] NSMomentum
	\item[] NSTemperature
\end{description}

Equations involved:
\begin{equation}
\rho (\vec{v}.\nabla)\vec{v} + \nabla p - \rho g \check{j} - \mu \nabla^{2}\vec{v} = 0
\end{equation}
\begin{equation}
\nabla.\vec{v}=0
\end{equation}
\begin{equation}
\rho c_{p} (\vec{v}.\nabla)T - \nabla .(k\nabla T) = 0
\end{equation}

Boundary conditions:
\begin{description}[]
	\item[] $v_{x}('left')=0$, $v_{x}('bottom')=0$, and $v_{x}('top')=0$.
	\item[] $v_{y}('left')=0$ and $v_{y}('right')=0$.
	\item[] $p('bottom')=0$ and $p('top')=1$.
	\item[] $T('left')=10$, $T('right')=10$, and $T('top')=5$.
\end{description}

The results are presented in Figure \ref{fig:ns08}.
\begin{figure*}[!h]
	\centering
	\begin{subfigure}[t]{0.4\textwidth}
		\centering
		\includegraphics[width=\linewidth]{ns08/vel_y.png} 
		\caption{ux.}
		\label{fig:ns08-vel}
	\end{subfigure}
	\vspace{1cm}
	\begin{subfigure}[t]{0.4\textwidth}
		\centering
		\includegraphics[width=\linewidth]{ns08/temp.png}
		\caption{uy.}
		\label{fig:ns08-temp}
	\end{subfigure}
	\hfill
	\caption{Temperature.}
	\label{fig:ns08}
\end{figure*}

\section{N-S: Example 09}

This is still a simplified version of moose's INS kernels, but this time it includes everything for Incompressible N-S and also INSTemperature slightly modified.
I am not completely sure that the Jacobian off diagonal for the TemperatureVD (Viscous Dissipation) is 100$\%$ right.

Kernels involved:
\begin{description}[font=$\bullet$\scshape\bfseries]
	\item[] NSMass
	\item[] NSMomentumTimeDerivative
	\item[] NSMomentum
	\item[] MatINSTemperatureTimeDerivative
	\item[] NSTemperature
	\item[] NSTemperatureVD
\end{description}

Equations involved:
\begin{equation}
\rho (\vec{v}.\nabla)\vec{v} + \nabla p - \rho g \check{j} - \mu \nabla^{2}\vec{v} = 0
\end{equation}
\begin{equation}
\nabla.\vec{v}=0
\end{equation}
\begin{equation}
\rho c_{p} (\vec{v}.\nabla)T - \nabla .(k\nabla T) - \Phi = 0
\end{equation}
\begin{equation}
\Phi = 2\mu((\frac{du}{dx})^2+(\frac{dv}{dy})^2+(\frac{dw}{dz})^2)+
\mu ((\frac{du}{dy}+\frac{dv}{dx})^2+(\frac{du}{dz}+\frac{dw}{dx})^2+(\frac{dv}{dz}+\frac{dw}{dy})^2)
\end{equation}

Boundary conditions:
\begin{description}[]
	\item[] $v_{x}('left')=0$, $v_{x}('bottom')=0$, and $v_{x}('top')=0$.
	\item[] $v_{y}('left')=0$ and $v_{y}('right')=0$.
	\item[] $p('bottom')=0$ and $p('top')=2000$.
	\item[] $T('left')=100$, $T('right')=100$, and $T('top')=0$.
\end{description}

The results are presented in Figure \ref{fig:ns09}. The blue line it the case that takes into account the viscous dissipation.
\begin{figure*}[!h]
	\centering
	\includegraphics[width=\linewidth]{ns09/temp.png} 
	\hfill
	\caption{Temperature ignoring and taking into account the viscous dissipation.}
	\label{fig:ns09}
\end{figure*}



\pagebreak 
\bibliographystyle{plain}
\bibliography{bibliography}

\end{document}