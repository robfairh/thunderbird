\documentclass[11pt,letterpaper]{article}
\usepackage[utf8]{inputenc}
\usepackage{caption} % for table captions
\usepackage{amsmath} % for multi-line equations and piecewises
\DeclareMathOperator{\sign}{sign}
\usepackage{graphicx}
\usepackage{relsize}
\usepackage{xspace}
\usepackage{float} 
\usepackage{verbatim} % for block comments
\usepackage{subcaption} % for subfigures
\usepackage{enumitem} % for a) b) c) lists
\newcommand{\Cyclus}{\textsc{Cyclus}\xspace}%
\newcommand{\Cycamore}{\textsc{Cycamore}\xspace}%
\newcommand{\deploy}{\texttt{d3ploy}\xspace}%
\newcommand{\Deploy}{\texttt{D3ploy}\xspace}%
\usepackage{tabularx}
\usepackage{color}
\usepackage{multirow}
\usepackage[acronym,toc]{glossaries}
\include{acros}
\definecolor{bg}{rgb}{0.95,0.95,0.95}
\newcolumntype{b}{X}
\newcolumntype{f}{>{\hsize=.15\hsize}X}
\newcolumntype{s}{>{\hsize=.5\hsize}X}
\newcolumntype{m}{>{\hsize=.75\hsize}X}
\newcolumntype{r}{>{\hsize=1.1\hsize}X}
\usepackage{titling}
\usepackage[hang,flushmargin]{footmisc}
\renewcommand*\footnoterule{}
\usepackage{tikz}

\usetikzlibrary{shapes.geometric,arrows}
\tikzstyle{process} = [rectangle, rounded corners, 
minimum width=1cm, minimum height=1cm,text centered, draw=black, 
fill=blue!30]
\tikzstyle{arrow} = [thick,->,>=stealth]

\graphicspath{}
%\title{Thermo-Mechanical MOOSE-based solver}
%\author{Roberto E. Fairhurst Agosta}

\begin{document}
	%\begin{titlepage}
		%\maketitle
		%\thispagestyle{empty}
	%\end{titlepage}
	
\section{ST00}

Eqs:
\begin{equation}
\frac{\partial \sigma_x}{\partial x} + \rho g = 0
\end{equation}
BCs:
\begin{equation}
\sigma_x (0) = 0
\end{equation}

\begin{figure}[H]
	\centering
	\includegraphics[width=\linewidth]{st00/sigma_x.png}
	\hfill
	\caption{$\sigma_x$.}
	\label{fig:st00}
\end{figure}

\section{ST01}

Eqs:
\begin{equation}
\frac{\partial \sigma_x}{\partial x} + \rho g = 0
\end{equation}
AuxKernel:
\begin{equation}
\varepsilon_x = \frac{1}{E}\sigma_x
\end{equation}
BCs:
\begin{equation}
\sigma_x (0) = 0
\end{equation}

\begin{figure}[H]
	\centering
	\includegraphics[width=\linewidth]{st01/strain_x.png}
	\hfill
	\caption{$\varepsilon_x$.}
	\label{fig:st01}
\end{figure}

\section{ST02}

Eqs:
\begin{equation}
\frac{\partial \sigma_x}{\partial x} + \rho g = 0
\end{equation}
\begin{equation}
\varepsilon_x = \frac{\partial u}{\partial x}
\end{equation}

AuxKernel:
\begin{equation}
\varepsilon_x = \frac{1}{E}\sigma_x
\end{equation}

BCs:
\begin{equation}
\sigma_x (0) = 0
\end{equation}
\begin{equation}
u (0) = 0
\end{equation}

Data: $\rho = 1700 kg/m3, E = 10 GPa, \nu = 0.14$

\begin{figure}[H]
	\centering
	\includegraphics[width=\linewidth]{st02/u.png}
	\hfill
	\caption{u.}
	\label{fig:st02}
\end{figure}

There is a problem here. If I try to define the following BCs it works.
BCs:
\begin{equation}
\sigma_x (0) = 0
\end{equation}
\begin{equation}
u (0) = 0
\end{equation}

Data: $\rho = 500 kg/m3, E = 10 GPa, \nu = 0.14$

Results:
$ \sigma_x (L) = -9.81e3$
$ u(L) = -9.81e-7$

Then if I try the following BCs, it doesn't run.
\begin{equation}
\sigma_x (L) = -9.81e3
\end{equation}
\begin{equation}
u (0) = 0
\end{equation}

\section{ST03}

Adds components to the previous case, just for the deformation equation.

\section{ST04}

1-D case.

Equilibrium eq:
\begin{equation}
\frac{\partial \sigma_x}{\partial x} + G_x = 0
\end{equation}
Constitutive eq:
\begin{equation}
\varepsilon_x = \frac{1}{E} \sigma_x
\end{equation}
Deformation equation:
\begin{equation}
\varepsilon_x = \frac{\partial u}{\partial x}
\end{equation}

Combining the eqs:
\begin{equation}
\frac{\partial}{\partial x}(E \frac{\partial u}{\partial x}) + G_x = 0
\end{equation}

In kernel form:
\begin{equation}
\left< E \frac{\partial u}{\partial x}, \phi \right>_{BC} - \left< E \frac{\partial u}{\partial x}, \frac{\partial \phi}{\partial x} \right>
+ \left< G_x , \phi \right> = 0
\end{equation}
Which is equal to:
\begin{equation}
-\left< -\sigma_x, \phi \right>_{BC} - \left< E \frac{\partial u}{\partial x}, \frac{\partial \phi}{\partial x} \right>
+ \left< G_x , \phi \right> = 0
\end{equation}
The last equation adds two minus signs to the BC term, as the Neumann BC adds a minus sign.

BCs: $u(0)=0, \sigma_x(L)=-9.81e3$

\begin{figure}[H]
	\centering
	\includegraphics[width=0.6\linewidth]{st04/u.png}
	\hfill
	\caption{u.}
	\label{fig:st04}
\end{figure}

I am not sure if the BC are properly implemented. A way to check is looking at the stress $\sigma_x$.
I was trying to implement an auxkernel that computes $E \frac{\partial u}{\partial x}$, but it throws an error.

\section{ST05}

2-D case.
Equilibrium eqs:
\begin{equation}
\frac{\partial \sigma_x}{\partial x} + \frac{\partial \tau_{xy}}{\partial y} + G_x = 0
\end{equation}
\begin{equation}
\frac{\partial \tau_{xy}}{\partial x} + \frac{\partial \sigma_y}{\partial y} + G_y = 0
\end{equation}
Constitutive eqs:
\begin{equation}
\varepsilon_x = \frac{1}{E} (\sigma_x - \nu \sigma_y)
\end{equation}
\begin{equation}
\varepsilon_y = \frac{1}{E} (\sigma_y - \nu \sigma_x)
\end{equation}
Deformation eqs:
\begin{equation}
\varepsilon_x = \frac{\partial u}{\partial x}
\end{equation}
\begin{equation}
\varepsilon_y = \frac{\partial v}{\partial y}
\end{equation}
\begin{equation}
\gamma_{xy} = \frac{\partial v}{\partial x} + \frac{\partial u}{\partial y}
\end{equation}

Combining the eqs:
\begin{equation}
\frac{\partial}{\partial x}(c \frac{\partial u}{\partial x} + d \frac{\partial v}{\partial y}) + \frac{\partial}{\partial y}(G \frac{\partial v}{\partial x} + G \frac{\partial u}{\partial y}) + G_x = 0
\end{equation}
\begin{equation}
\frac{\partial}{\partial y}(G \frac{\partial v}{\partial x} + G \frac{\partial u}{\partial y}) + \frac{\partial}{\partial y}(d \frac{\partial u}{\partial x} + c \frac{\partial v}{\partial y}) + G_y = 0
\end{equation}
where $c = \frac{E}{1-\nu^2}$, $d = \frac{\nu E}{1-\nu^2}$, and $G = \frac{E}{2(1+\nu)}$.

In kernel form:
\begin{equation}
\left< c \frac{\partial u}{\partial x} + d \frac{\partial v}{\partial y}, \phi \right>_{BC_x} - \left< c \frac{\partial u}{\partial x} + d \frac{\partial v}{\partial y}, \frac{\partial \phi}{\partial x} \right> + \left< G \frac{\partial v}{\partial x} + G \frac{\partial u}{\partial y}, \phi \right>_{BC_y} - \left< G \frac{\partial v}{\partial x} + G \frac{\partial u}{\partial y}, \frac{\partial \phi}{\partial y} \right>
+ \left< G_x , \phi \right> = 0
\end{equation}
\begin{equation}
\left< G \frac{\partial v}{\partial x} + G \frac{\partial u}{\partial y}, \phi \right>_{BC_x} - \left< G \frac{\partial v}{\partial x} + G \frac{\partial u}{\partial y}, \frac{\partial \phi}{\partial x} \right> +
\left< d \frac{\partial u}{\partial x} + c \frac{\partial v}{\partial y}, \phi \right>_{BC_y} - \left< d \frac{\partial u}{\partial x} + c \frac{\partial v}{\partial y}, \frac{\partial \phi}{\partial y} \right>
+ \left< G_y , \phi \right> = 0
\end{equation}

\section{Meeting 09/19}

2-D case.
Equilibrium eqs:
\begin{equation}
\frac{\partial \sigma_x}{\partial x} + \frac{\partial \tau_{xy}}{\partial y} + G_x = 0
\end{equation}
\begin{equation}
\frac{\partial \tau_{xy}}{\partial x} + \frac{\partial \sigma_y}{\partial y} + G_y = 0
\end{equation}
Constitutive eqs:
\begin{equation}
\varepsilon_x = \frac{1}{E} (\sigma_x - \nu \sigma_y) + \alpha (T-T_{ref}) + \varepsilon_{Rad-Induced}
\end{equation}
\begin{equation}
\varepsilon_y = \frac{1}{E} (\sigma_y - \nu \sigma_x) + \alpha (T-T_{ref}) + \varepsilon_{Rad-Induced}
\end{equation}

\begin{equation}
\varepsilon_{Rad-Induced} = A \Phi^2 + B \Phi
\end{equation}

Deformation eqs:
\begin{equation}
\varepsilon_x = \frac{\partial u}{\partial x}
\end{equation}
\begin{equation}
\varepsilon_y = \frac{\partial v}{\partial y}
\end{equation}
\begin{equation}
\gamma_{xy} = \frac{\partial v}{\partial x} + \frac{\partial u}{\partial y}
\end{equation}

\pagebreak 
\bibliographystyle{plain}
\bibliography{bibliography}

\end{document}
