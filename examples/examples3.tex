\documentclass[11pt,letterpaper]{article}
\usepackage[utf8]{inputenc}
\usepackage{caption} % for table captions
\usepackage{amsmath} % for multi-line equations and piecewises
\DeclareMathOperator{\sign}{sign}
\usepackage{graphicx}
\usepackage{relsize}
\usepackage{xspace}
\usepackage{float} 
\usepackage{verbatim} % for block comments
\usepackage{subcaption} % for subfigures
\usepackage{enumitem} % for a) b) c) lists
\newcommand{\Cyclus}{\textsc{Cyclus}\xspace}%
\newcommand{\Cycamore}{\textsc{Cycamore}\xspace}%
\newcommand{\deploy}{\texttt{d3ploy}\xspace}%
\newcommand{\Deploy}{\texttt{D3ploy}\xspace}%
\usepackage{tabularx}
\usepackage{color}
\usepackage{multirow}
\usepackage[acronym,toc]{glossaries}
\newacronym[longplural={metric tons of heavy metal}]{MTHM}{MTHM}{metric ton of heavy metal}
\newacronym{ABM}{ABM}{agent-based modeling}
\newacronym{ACDIS}{ACDIS}{Program in Arms Control \& Domestic and International Security}
\newacronym{AHTR}{AHTR}{Advanced High Temperature Reactor}
\newacronym{ANDRA}{ANDRA}{Agence Nationale pour la gestion des D\'echets RAdioactifs, the French National Agency for Radioactive Waste Management}
\newacronym{ANL}{ANL}{Argonne National Laboratory}
\newacronym{API}{API}{application programming interface}
\newacronym{ARCH}{ARCH}{autoregressive conditional heteroskedastic}
\newacronym{ARE}{ARE}{Aircraft Reactor Experiment}
\newacronym{ARFC}{ARFC}{Advanced Reactors and Fuel Cycles}
\newacronym{ARMA}{ARMA}{autoregressive moving average}
\newacronym{ASME}{ASME}{American Society of Mechanical Engineers}
\newacronym{ATWS}{ATWS}{Anticipated Transient Without Scram}
\newacronym{BDBE}{BDBE}{Beyond Design Basis Event}
\newacronym{BIDS}{BIDS}{Berkeley Institute for Data Science}
\newacronym{BOL}{BOL}{Beginning-of-Life}
\newacronym{BSD}{BSD}{Berkeley Software Distribution}
\newacronym{CAFCA}{CAFCA}{ Code for Advanced Fuel Cycles Assessment }
\newacronym{CASL}{CASL}{Consortium for Advanced Simulation of Light Water Reactors}
\newacronym{CDTN}{CDTN}{Centro de Desenvolvimento da Tecnologia Nuclear}
\newacronym{CEA}{CEA}{Commissariat \`a l'\'Energie Atomique et aux \'Energies Alternatives}
\newacronym{CI}{CI}{continuous integration}
\newacronym{CNEC}{CNEC}{Consortium for Nonproliferation Enabling Capabilities}
\newacronym{CNEN}{CNEN}{Comiss\~{a}o Nacional de Energia Nuclear}
\newacronym{CNERG}{CNERG}{Computational Nuclear Engineering Research Group}
\newacronym{COSI}{COSI}{Commelini-Sicard}
\newacronym{COTS}{COTS}{commercial, off-the-shelf}
\newacronym{CSNF}{CSNF}{commercial spent nuclear fuel}
\newacronym{CTAH}{CTAHs}{Coiled Tube Air Heaters}
\newacronym{CUBIT}{CUBIT}{CUBIT Geometry and Mesh Generation Toolkit}
\newacronym{CURIE}{CURIE}{Centralized Used Fuel Resource for Information Exchange}
\newacronym{DAG}{DAG}{directed acyclic graph}
\newacronym{DANESS}{DANESS}{Dynamic Analysis of Nuclear Energy System Strategies}
\newacronym{DBE}{DBE}{Design Basis Event}
\newacronym{DESAE}{DESAE}{Dynamic Analysis of Nuclear Energy Systems Strategies}
\newacronym{DHS}{DHS}{Department of Homeland Security}
\newacronym{DOE}{DOE}{Department of Energy}
\newacronym{DRACS}{DRACS}{Direct Reactor Auxiliary Cooling System}
\newacronym{DRE}{DRE}{dynamic resource exchange}
\newacronym{DSNF}{DSNF}{DOE spent nuclear fuel}
\newacronym{DYMOND}{DYMOND}{Dynamic Model of Nuclear Development }
\newacronym{EBS}{EBS}{Engineered Barrier System}
\newacronym{EDZ}{EDZ}{Excavation Disturbed Zone}
\newacronym{EIA}{EIA}{U.S. Energy Information Administration}
\newacronym{EPA}{EPA}{Environmental Protection Agency}
\newacronym{EP}{EP}{Engineering Physics}
\newacronym{FCO}{FCO}{Fuel Cycle Options}
\newacronym{FCT}{FCT}{Fuel Cycle Technology}
\newacronym{FCWMD}{FCWMD}{Fuel Cycle and Waste Management Division}
\newacronym{FEHM}{FEHM}{Finite Element Heat and Mass Transfer}
\newacronym{FEPs}{FEPs}{Features, Events, and Processes}
\newacronym{FHR}{FHR}{Fluoride-Salt-Cooled High-Temperature Reactor}
\newacronym{FLiBe}{FLiBe}{Fluoride-Lithium-Beryllium}
\newacronym{GCAM}{GCAM}{Global Change Assessment Model}
\newacronym{GDSE}{GDSE}{Generic Disposal System Environment}
\newacronym{GDSM}{GDSM}{Generic Disposal System Model}
\newacronym{GENIUSv1}{GENIUSv1}{Global Evaluation of Nuclear Infrastructure Utilization Scenarios, Version 1}
\newacronym{GENIUSv2}{GENIUSv2}{Global Evaluation of Nuclear Infrastructure Utilization Scenarios, Version 2}
\newacronym{GENIUS}{GENIUS}{Global Evaluation of Nuclear Infrastructure Utilization Scenarios}
\newacronym{GPAM}{GPAM}{Generic Performance Assessment Model}
\newacronym{GRSAC}{GRSAC}{Graphite Reactor Severe Accident Code}
\newacronym{GUI}{GUI}{graphical user interface}
\newacronym{HLW}{HLW}{high level waste}
\newacronym{HPC}{HPC}{high-performance computing}
\newacronym{HTC}{HTC}{high-throughput computing}
\newacronym{HTGR}{HTGR}{High Temperature Gas-Cooled Reactor}
\newacronym{IAEA}{IAEA}{International Atomic Energy Agency}
\newacronym{IEMA}{IEMA}{Illinois Emergency Mangament Agency}
\newacronym{INL}{INL}{Idaho National Laboratory}
\newacronym{IPRR1}{IRP-R1}{Instituto de Pesquisas Radioativas Reator 1}
\newacronym{IRP}{IRP}{Integrated Research Project}
\newacronym{ISFSI}{ISFSI}{Independent Spent Fuel Storage Installation}
\newacronym{ISRG}{ISRG}{Independent Student Research Group}
\newacronym{JFNK}{JFNK}{Jacobian-Free Newton Krylov}
\newacronym{LANL}{LANL}{Los Alamos National Laboratory}
\newacronym{LBNL}{LBNL}{Lawrence Berkeley National Laboratory}
\newacronym{LCOE}{LCOE}{levelized cost of electricity}
\newacronym{LDRD}{LDRD}{laboratory directed research and development}
\newacronym{LFR}{LFR}{Lead-Cooled Fast Reactor}
\newacronym{LGPL}{LGPL}{Lesser GNU Public License}
\newacronym{LLNL}{LLNL}{Lawrence Livermore National Laboratory}
\newacronym{LMFBR}{LMFBR}{Liquid-Metal-cooled Fast Breeder Reactor}
\newacronym{LOFC}{LOFC}{Loss of Forced Cooling}
\newacronym{LOHS}{LOHS}{Loss of Heat Sink}
\newacronym{LOLA}{LOLA}{Loss of Large Area}
\newacronym{LP}{LP}{linear program}
\newacronym{LWR}{LWR}{Light Water Reactor}
\newacronym{MARKAL}{MARKAL}{MARKet and ALlocation}
\newacronym{MA}{MA}{minor actinide}
\newacronym{MCNP}{MCNP}{Monte Carlo N-Particle code}
\newacronym{MILP}{MILP}{mixed-integer linear program}
\newacronym{MIT}{MIT}{the Massachusetts Institute of Technology}
\newacronym{MOAB}{MOAB}{Mesh-Oriented datABase}
\newacronym{MOOSE}{MOOSE}{Multiphysics Object-Oriented Simulation Environment}
\newacronym{MOX}{MOX}{mixed oxide}
\newacronym{MSBR}{MSBR}{Molten Salt Breeder Reactor}
\newacronym{MSRE}{MSRE}{Molten Salt Reactor Experiment}
\newacronym{MSR}{MSR}{Molten Salt Reactor}
\newacronym{NAGRA}{NAGRA}{National Cooperative for the Disposal of Radioactive Waste}
\newacronym{NCSA}{NCSA}{National Center for Supercomputing Applications}
\newacronym{NEAMS}{NEAMS}{Nuclear Engineering Advanced Modeling and Simulation}
\newacronym{NEUP}{NEUP}{Nuclear Energy University Programs}
\newacronym{NFCSim}{NFCSim}{Nuclear Fuel Cycle Simulator}
\newacronym{NFC}{NFC}{Nuclear Fuel Cycle}
\newacronym{NGNP}{NGNP}{Next Generation Nuclear Plant}
\newacronym{NMWPC}{NMWPC}{Nuclear MW Per Capita}
\newacronym{NNSA}{NNSA}{National Nuclear Security Administration}
\newacronym{NPRE}{NPRE}{Department of Nuclear, Plasma, and Radiological Engineering}
\newacronym{NQA1}{NQA-1}{Nuclear Quality Assurance - 1}
\newacronym{NRC}{NRC}{Nuclear Regulatory Commission}
\newacronym{NSF}{NSF}{National Science Foundation}
\newacronym{NSSC}{NSSC}{Nuclear Science and Security Consortium}
\newacronym{NUWASTE}{NUWASTE}{Nuclear Waste Assessment System for Technical Evaluation}
\newacronym{NWF}{NWF}{Nuclear Waste Fund}
\newacronym{NWTRB}{NWTRB}{Nuclear Waste Technical Review Board}
\newacronym{OCRWM}{OCRWM}{Office of Civilian Radioactive Waste Management}
\newacronym{ORION}{ORION}{ORION}
\newacronym{ORNL}{ORNL}{Oak Ridge National Laboratory}
\newacronym{PARCS}{PARCS}{Purdue Advanced Reactor Core Simulator}
\newacronym{PBAHTR}{PB-AHTR}{Pebble Bed Advanced High Temperature Reactor}
\newacronym{PBFHR}{PB-FHR}{Pebble-Bed Fluoride-Salt-Cooled High-Temperature Reactor}
\newacronym{PEI}{PEI}{Peak Environmental Impact}
\newacronym{PH}{PRONGHORN}{PRONGHORN}
\newacronym{PI}{PI}{Principal Investigator}
\newacronym{PNNL}{PNNL}{Pacific Northwest National Laboratory}
\newacronym{PRIS}{PRIS}{Power Reactor Information System}
\newacronym{PRKE}{PRKE}{Point Reactor Kinetics Equations}
\newacronym{PSPG}{PSPG}{Pressure-Stabilizing/Petrov-Galerkin}
\newacronym{PWAR}{PWAR}{Pratt and Whitney Aircraft Reactor}
\newacronym{PWR}{PWR}{Pressurized Water Reactor}
\newacronym{PyNE}{PyNE}{Python toolkit for Nuclear Engineering}
\newacronym{PyRK}{PyRK}{Python for Reactor Kinetics}
\newacronym{QA}{QA}{quality assurance}
\newacronym{RDD}{RD\&D}{Research Development and Demonstration}
\newacronym{RD}{R\&D}{Research and Development}
\newacronym{RELAP}{RELAP}{Reactor Excursion and Leak Analysis Program}
\newacronym{RIA}{RIA}{Reactivity Insertion Accident}
\newacronym{RIF}{RIF}{Region-Institution-Facility}
\newacronym{SAM}{SAM}{Simulation and Modeling}
\newacronym{SCF}{SCF}{Software Carpentry Foundation}
\newacronym{SFR}{SFR}{Sodium-Cooled Fast Reactor}
\newacronym{SINDAG}{SINDA{\textbackslash}G}{Systems Improved Numerical Differencing Analyzer $\backslash$ Gaski}
\newacronym{SKB}{SKB}{Svensk K\"{a}rnbr\"{a}nslehantering AB}
\newacronym{SNF}{SNF}{spent nuclear fuel}
\newacronym{SNL}{SNL}{Sandia National Laboratory}
\newacronym{SNM}{SNM}{Special Nuclear Material}
\newacronym{STC}{STC}{specific temperature change}
\newacronym{SUPG}{SUPG}{Streamline-Upwind/Petrov-Galerkin}
\newacronym{SWF}{SWF}{Separations and Waste Forms}
\newacronym{SWU}{SWU}{Separative Work Unit}
\newacronym{SandO}{S\&O}{Signatures and Observables}
\newacronym{THW}{THW}{The Hacker Within}
\newacronym{TRIGA}{TRIGA}{Training Research Isotope General Atomic}
\newacronym{TRISO}{TRISO}{Tristructural Isotropic}
\newacronym{TSM}{TSM}{Total System Model}
\newacronym{TSPA}{TSPA}{Total System Performance Assessment for the Yucca Mountain License Application}
\newacronym{UDB}{UDB}{Unified Database}
\newacronym{UFD}{UFD}{Used Fuel Disposition}
\newacronym{UML}{UML}{Unified Modeling Language}
\newacronym{UNFSTANDARDS}{UNFST\&DARDS}{Used Nuclear Fuel Storage, Transportation \& Disposal Analysis Resource and Data System}
\newacronym{UOX}{UOX}{uranium oxide}
\newacronym{UQ}{UQ}{uncertainty quantification}
\newacronym{US}{US}{United States}
\newacronym{UW}{UW}{University of Wisconsin}
\newacronym{VISION}{VISION}{the Verifiable Fuel Cycle Simulation Model}
\newacronym{VV}{V\&V}{verification and validation}
\newacronym{WIPP}{WIPP}{Waste Isolation Pilot Plant}
\newacronym{YMG}{YMG}{Young Members Group}
\newacronym{YMR}{YMR}{Yucca Mountain Repository Site}
\newacronym{NEI}{NEI}{Nuclear Energy Institute}
%\newacronym{<++>}{<++>}{<++>}
%\newacronym{<++>}{<++>}{<++>}

\definecolor{bg}{rgb}{0.95,0.95,0.95}
\newcolumntype{b}{X}
\newcolumntype{f}{>{\hsize=.15\hsize}X}
\newcolumntype{s}{>{\hsize=.5\hsize}X}
\newcolumntype{m}{>{\hsize=.75\hsize}X}
\newcolumntype{r}{>{\hsize=1.1\hsize}X}
\usepackage{titling}
\usepackage[hang,flushmargin]{footmisc}
\renewcommand*\footnoterule{}
\usepackage{tikz}

\usetikzlibrary{shapes.geometric,arrows}
\tikzstyle{process} = [rectangle, rounded corners, 
minimum width=1cm, minimum height=1cm,text centered, draw=black, 
fill=blue!30]
\tikzstyle{arrow} = [thick,->,>=stealth]

\graphicspath{}
%\title{Thermo-Mechanical MOOSE-based solver}
%\author{Roberto E. Fairhurst Agosta}

\begin{document}
	%\begin{titlepage}
		%\maketitle
		%\thispagestyle{empty}
	%\end{titlepage}
	
\section{ST00}

Eqs:
\begin{equation}
\frac{\partial \sigma_x}{\partial x} + \rho g = 0
\end{equation}
BCs:
\begin{equation}
\sigma_x (0) = 0
\end{equation}

\begin{figure}[H]
	\centering
	\includegraphics[width=\linewidth]{st00/sigma_x.png}
	\hfill
	\caption{$\sigma_x$.}
	\label{fig:st00}
\end{figure}

\section{ST01}

Eqs:
\begin{equation}
\frac{\partial \sigma_x}{\partial x} + \rho g = 0
\end{equation}
AuxKernel:
\begin{equation}
\varepsilon_x = \frac{1}{E}\sigma_x
\end{equation}
BCs:
\begin{equation}
\sigma_x (0) = 0
\end{equation}

\begin{figure}[H]
	\centering
	\includegraphics[width=\linewidth]{st01/strain_x.png}
	\hfill
	\caption{$\varepsilon_x$.}
	\label{fig:st01}
\end{figure}

\section{ST02}

Eqs:
\begin{equation}
\frac{\partial \sigma_x}{\partial x} + \rho g = 0
\end{equation}
\begin{equation}
\varepsilon_x = \frac{\partial u}{\partial x}
\end{equation}

AuxKernel:
\begin{equation}
\varepsilon_x = \frac{1}{E}\sigma_x
\end{equation}

BCs:
\begin{equation}
\sigma_x (0) = 0
\end{equation}
\begin{equation}
u (0) = 0
\end{equation}

Data: $\rho = 1700 kg/m3, E = 10 GPa, \nu = 0.14$

\begin{figure}[H]
	\centering
	\includegraphics[width=\linewidth]{st02/u.png}
	\hfill
	\caption{u.}
	\label{fig:st02}
\end{figure}

There is a problem here. If I try to define the following BCs it works.
BCs:
\begin{equation}
\sigma_x (0) = 0
\end{equation}
\begin{equation}
u (0) = 0
\end{equation}

Data: $\rho = 500 kg/m3, E = 10 GPa, \nu = 0.14$

Results:
$ \sigma_x (L) = -9.81e3$
$ u(L) = -9.81e-7$

Then if I try the following BCs, it doesn't run.
\begin{equation}
\sigma_x (L) = -9.81e3
\end{equation}
\begin{equation}
u (0) = 0
\end{equation}

\section{ST03}

2-D case. See Hooke's Law. $\sigma_x$ and $\sigma_y$ are independent.

Equilibrium eq:
\begin{equation}
\frac{\partial \sigma_x}{\partial x} = 0
\end{equation}
\begin{equation}
\frac{\partial \sigma_y}{\partial y} + \rho * g = 0
\end{equation}

Hooke's Law:
\begin{equation}
\varepsilon_x = \frac{1}{E} \sigma_x
\end{equation}
\begin{equation}
\varepsilon_y = \frac{1}{E} \sigma_y
\end{equation}

Deformation equation:
\begin{equation}
\varepsilon_x = \frac{\partial u}{\partial x}
\end{equation}
\begin{equation}
\varepsilon_y = \frac{\partial v}{\partial y}
\end{equation}

BCs:
\begin{equation}
\sigma_x (0, y) = 0
\end{equation}
\begin{equation}
\sigma_y (x, 0) = 0
\end{equation}

\begin{equation}
u (0, y) = 0
\end{equation}
\begin{equation}
v (x, 0) = 0
\end{equation}

\section{ST04}

1-D case.

Equilibrium eq:
\begin{equation}
\frac{\partial \sigma_x}{\partial x} + G_x = 0
\end{equation}
Constitutive eq:
\begin{equation}
\varepsilon_x = \frac{1}{E} \sigma_x
\end{equation}
Deformation equation:
\begin{equation}
\varepsilon_x = \frac{\partial u}{\partial x}
\end{equation}

Combining the eqs:
\begin{equation}
\frac{\partial}{\partial x}(E \frac{\partial u}{\partial x}) + G_x = 0
\end{equation}

In kernel form:
\begin{equation}
\left< E \frac{\partial u}{\partial x}, \phi \right>_{BC} - \left< E \frac{\partial u}{\partial x}, \frac{\partial \phi}{\partial x} \right>
+ \left< G_x , \phi \right> = 0
\end{equation}
Which is equal to:
\begin{equation}
-\left< -\sigma_x, \phi \right>_{BC} - \left< E \frac{\partial u}{\partial x}, \frac{\partial \phi}{\partial x} \right>
+ \left< G_x , \phi \right> = 0
\end{equation}
The last equation adds two minus signs to the BC term, as the Neumann BC adds a minus sign.

BCs: $u(0)=0, \sigma_x(L)=-9.81e3$

\begin{figure}[H]
	\centering
	\includegraphics[width=0.6\linewidth]{st04/u.png}
	\hfill
	\caption{u.}
	\label{fig:st04}
\end{figure}

I am not sure if the BC are properly implemented. A way to check is looking at the stress $\sigma_x$.
I was trying to implement an auxkernel that computes $E \frac{\partial u}{\partial x}$, but it throws an error.

\section{ST05}

2-D case.
Equilibrium eqs:
\begin{equation}
\frac{\partial \sigma_x}{\partial x} + \frac{\partial \tau_{xy}}{\partial y} + G_x = 0
\end{equation}
\begin{equation}
\frac{\partial \tau_{xy}}{\partial x} + \frac{\partial \sigma_y}{\partial y} + G_y = 0
\end{equation}
Constitutive eqs:
\begin{equation}
\varepsilon_x = \frac{1}{E} (\sigma_x - \nu \sigma_y)
\end{equation}
\begin{equation}
\varepsilon_y = \frac{1}{E} (\sigma_y - \nu \sigma_x)
\end{equation}
Deformation eqs:
\begin{equation}
\varepsilon_x = \frac{\partial u}{\partial x}
\end{equation}
\begin{equation}
\varepsilon_y = \frac{\partial v}{\partial y}
\end{equation}
\begin{equation}
\gamma_{xy} = \frac{\partial v}{\partial x} + \frac{\partial u}{\partial y}
\end{equation}

Combining the eqs:
\begin{equation}
\frac{\partial}{\partial x}(c \frac{\partial u}{\partial x} + d \frac{\partial v}{\partial y}) + \frac{\partial}{\partial y}(G \frac{\partial v}{\partial x} + G \frac{\partial u}{\partial y}) + G_x = 0
\end{equation}
\begin{equation}
\frac{\partial}{\partial y}(G \frac{\partial v}{\partial x} + G \frac{\partial u}{\partial y}) + \frac{\partial}{\partial y}(d \frac{\partial u}{\partial x} + c \frac{\partial v}{\partial y}) + G_y = 0
\end{equation}
where $c = \frac{E}{1-\nu^2}$, $d = \frac{\nu E}{1-\nu^2}$, and $G = \frac{E}{2(1+\nu)}$.

In kernel form:
\begin{equation}
\left< c \frac{\partial u}{\partial x} + d \frac{\partial v}{\partial y}, \phi \right>_{BC_x} - \left< c \frac{\partial u}{\partial x} + d \frac{\partial v}{\partial y}, \frac{\partial \phi}{\partial x} \right> + \left< G \frac{\partial v}{\partial x} + G \frac{\partial u}{\partial y}, \phi \right>_{BC_y} - \left< G \frac{\partial v}{\partial x} + G \frac{\partial u}{\partial y}, \frac{\partial \phi}{\partial y} \right>
+ \left< G_x , \phi \right> = 0
\end{equation}
\begin{equation}
\left< G \frac{\partial v}{\partial x} + G \frac{\partial u}{\partial y}, \phi \right>_{BC_x} - \left< G \frac{\partial v}{\partial x} + G \frac{\partial u}{\partial y}, \frac{\partial \phi}{\partial x} \right> +
\left< d \frac{\partial u}{\partial x} + c \frac{\partial v}{\partial y}, \phi \right>_{BC_y} - \left< d \frac{\partial u}{\partial x} + c \frac{\partial v}{\partial y}, \frac{\partial \phi}{\partial y} \right>
+ \left< G_y , \phi \right> = 0
\end{equation}

I don't know how to impose the boundary condition here. Let's go back to the case ST03.
$\sigma_x = c\frac{\partial u}{\partial x} + d\frac{\partial v}{\partial y}$
Try with Neumann, couple the variables u and v and replace value by my $\sigma_x$.

\section{Meeting 09/19}

2-D case.
Equilibrium eqs:
\begin{equation}
\frac{\partial \sigma_x}{\partial x} + \frac{\partial \tau_{xy}}{\partial y} + G_x = 0
\end{equation}
\begin{equation}
\frac{\partial \tau_{xy}}{\partial x} + \frac{\partial \sigma_y}{\partial y} + G_y = 0
\end{equation}
Constitutive eqs:
\begin{equation}
\varepsilon_x = \frac{1}{E} (\sigma_x - \nu \sigma_y) + \alpha (T-T_{ref}) + \varepsilon_{Rad-Induced}
\end{equation}
\begin{equation}
\varepsilon_y = \frac{1}{E} (\sigma_y - \nu \sigma_x) + \alpha (T-T_{ref}) + \varepsilon_{Rad-Induced}
\end{equation}

\begin{equation}
\varepsilon_{Rad-Induced} = A \Phi^2 + B \Phi
\end{equation}

Deformation eqs:
\begin{equation}
\varepsilon_x = \frac{\partial u}{\partial x}
\end{equation}
\begin{equation}
\varepsilon_y = \frac{\partial v}{\partial y}
\end{equation}
\begin{equation}
\gamma_{xy} = \frac{\partial v}{\partial x} + \frac{\partial u}{\partial y}
\end{equation}

\section{ST06}

2-D case.

Equilibrium eq:
\begin{equation}
\frac{\partial \sigma_x}{\partial x} = 0
\end{equation}
\begin{equation}
\frac{\partial \sigma_y}{\partial y} + \rho * g = 0
\end{equation}

Constitutive eq:
\begin{equation}
\varepsilon_x = \frac{1}{E} \sigma_x
\end{equation}
\begin{equation}
\varepsilon_y = \frac{1}{E} \sigma_y
\end{equation}

Deformation equation:
\begin{equation}
\varepsilon_x = \frac{\partial u}{\partial x}
\end{equation}
\begin{equation}
\varepsilon_y = \frac{\partial v}{\partial y}
\end{equation}

BCs:
\begin{equation}
\sigma_x ('center', y) = 0
\end{equation}
\begin{equation}
\sigma_y (x, 'top') = 0
\end{equation}

\begin{equation}
u ('center', y) = 0
\end{equation}
\begin{equation}
v (x, 'bottom') = 0
\end{equation}

Doesn't work, idk why. It does something weird with the BCs.

\section{ST07}

\begin{equation}
\frac{\partial \sigma_x}{\partial x} + \rho * g_x = 0
\end{equation}

Multiply by $\phi$:
\begin{equation}
\frac{\partial \sigma_x}{\partial x} \phi + \rho g_x \phi = 0
\end{equation}
Using $\frac{\partial \sigma_x \phi}{\partial x}=\frac{\partial \sigma_x}{\partial x} \phi + \sigma_x \frac{\partial \phi}{\partial x}$
\begin{equation}
\frac{\partial \sigma_x \phi}{\partial x} - \sigma_x \frac{\partial \phi}{\partial x} + \rho g_x \phi = 0
\end{equation}
Integrating over the Volume:
\begin{equation}
\left< \frac{\partial \sigma_x \phi}{\partial x} \right> - \left< \sigma_x,\frac{\partial \phi}{\partial x} \right> + \left< \rho g_x,\phi \right> = 0
\end{equation}
\begin{equation}
\left< \sigma_x \phi \right>_{BC} - \left< \sigma_x,\frac{\partial \phi}{\partial x} \right> + \left< \rho g_x,\phi \right> = 0
\end{equation}
It doesn't work. Super weird ...

\begin{figure}[H]
	\centering
	\includegraphics[width=0.6\linewidth]{st07/sigma_y.png}
	\hfill
	\caption{$\sigma_y$.}
	\label{fig:st07}
\end{figure}

\section{ST08}

Try to implement the module: Tensor-mechanics from Moose.



\pagebreak 
\bibliographystyle{plain}
\bibliography{bibliography}

\end{document}
